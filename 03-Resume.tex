\begingroup
\fontsize{10pt}{10pt}\selectfont

\begin{center}
\textbf{Résumé}
\end{center}

\par
Ce rapport aborde le projet final réalisé chez Neoledge à Zarzis pour l'obtention d'une licence en technologie de l'informatique. Le projet était axé sur le développement d'une application GED mobile appelée Elise Mobile qui a été développée en utilisant Ionic 6, Vue.js 3, Capacitor 4 et TypeScript. \\
Les principales fonctionnalités d'Elise Mobile comprennent la gestion des absences, l'accès à différents espaces de travail, la gestion de documents, la gestion de fichiers ainsi que les signatures, y compris la signature électronique des documents. \\
Le rapport fournit des informations détaillées sur la conception, le développement et la mise en œuvre d'Elise Mobile, ainsi que sur les choix techniques et les défis rencontrés. Il met également en évidence les résultats obtenus et les perspectives d'amélioration future.\\
\noindent\rule[2pt]{\textwidth}{0.5pt}
{\textbf{Mot clé :}}
OIDC, GED, SSO, UI, UX, Mobile, Ionic, Capacitor, Vue.js, TypeScript, Notification push, Firebase, Android, Ios
\\
\noindent\rule[2pt]{\textwidth}{0.5pt}

\begin{center}
\textbf{Abstract}
\end{center}
\par
This report discusses the final project at Neoledge in Zarzis. The project focused on developing a mobile GED application called Elise Mobile that was developed using Ionic 6, Vue.js 3, Capacitor 4, and TypeScript. \\
Key features of Elise Mobile include absence management, access to different workspace areas, document management, file management, and signatures, including electronic document signing.\\
The report provides detailed information on the design, development, and implementation of Elise Mobile, as well as technical choices and challenges. It also highlights achieved results and future improvement prospects.

\noindent\rule[2pt]{\textwidth}{0.5pt}

{\textbf{Keywords :}}
OIDC, GED, SSO, UI, UX, Mobile, Ionic, Capacitor, Vue.js, TypeScript, Notification push, Firebase, Android, Ios
\\
\noindent\rule[2pt]{\textwidth}{0.5pt}

\begin{center}
  \begin{Arabic}
    \textbf{ملخص}
  \end{Arabic}
\end{center}
% arabic toc


\begin{Arabic}
  \par
  يتناول هذا التقرير المشروع النهائي في نيوليدج, جرجيس. ركز المشروع على تطوير تطبيق جوال لإدارة المستندات الإلكترونية المعروف بـ "إليز موبايل".\\
تم تطوير إليز موبايل باستخدام تقنيات ايونيك 6 و فيو جي اس 3 و كاباسيتور 4 و تايب سكريبت.
تشمل الميزات الرئيسية لإليز موبايل إدارة الغيابات والوصول إلى مساحات العمل المختلفة وإدارة المستندات وإدارة الملفات والتوقيعات، بما في ذلك توقيع المستندات الإلكترونية.\\
يقدم التقرير معلومات مفصلة حول التصميم والتطوير والتنفيذ لتطبيق إليز موبايل، بالإضافة إلى الخيارات التقنية والتحديات التي واجهناها. كما يسلط الضوء على النتائج التي تم تحقيقها وآفاق التحسين المستقبلية.
  
\end{Arabic}

\noindent\rule[2pt]{\textwidth}{0.5pt}

\begin{Arabic}

{\textbf{كلمات مفتاحية :}}
او اي دي سي, جيد, يو آي, يو اكس, موبايل, ايونيك, كاباسيتور, فيو جي اس, تايب سكريبة, اشعارات, فاير بايز, اندرويد, اي او اس
\end{Arabic}
\\
\noindent\rule[2pt]{\textwidth}{0.5pt}

\endgroup