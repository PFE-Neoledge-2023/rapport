\chapter*{Introduction générale}
\addcontentsline{toc}{chapter}{Introduction générale}
\markboth{Introduction générale}{Introduction générale}
\label{chap:introduction}
%\minitoc

Un organisme public ou privé acquiert et produit tout au long de son activité un grand nombre de documents. Certains sont vitaux (les titres de propriété, les contrats, etc) et doivent être conservés pour répondre à l'environnement réglementaire. D'autres encore, les documents dits « de travail » tels que les comptes-rendus, les rapports, les documents bureautiques, peuvent être consultés dans le but de prendre une décision. Par conséquent, la gestion et la conservation des documents au sein de l'organisme sont des activités essentielles, elles l'offrent un gain de temps et d'argent et une centralisation des documents et sécurisation des données ce qui est offert par le système de gestion électronique du document GED.
\medskip

Pour cela les logiciels de GED est devenu aujourd'hui la meilleure solution pour faire la gestion électronique des documents avec un minimum d'effort.
\medskip

C'est dans ce cadre que se situe notre projet de fin d'études effectué au sein de la société NeoLedge pendant la période de 06/02/2022 au 27/05/2022 pour l'obtention du diplôme de licence national en technologie d'information.
\medskip

Il s'agit en fait de concevoir et développer une application mobile « Cross-Platform » Permettant d'offrir aux utilisateurs de l'application « Elise Web » les mêmes fonctionnalités. Cette solution permettra la gestion et la conservation des documents qui sont des activités essentielles, de plus elle permet de faciliter le travail collaboratif.

\medskip
Le présent rapport synthétise ainsi le déroulement de notre travail sur ce projet. Il est structuré en cinq chapitres :

\medskip
Le premier chapitre sera consacré au contexte général du projet. Nous allons, tout d'abord, Présenter la société dans laquelle a été réalisée notre application, ainsi, nous allons définir notre mission et les objectifs à atteindre avec l'analyse et la critique de l'existant. Ensuite, nous allons écrire la méthodologie de développement adoptée et nos choix pour la modélisation conceptuelle. 

\medskip
Le deuxième chapitre présentera l'analyse et la conception en détail répondant aux exigences fonctionnelles et non fonctionnelles en se basant sur la méthodologie SCRUM.

\medskip
Le troisième chapitre présentera les sprint 1 \& 2 « Gestion d'authentification » \& « Gestion de profil ». Nous allons, tout d'abord, présenter pour chaque sprint le sprint Backlog et le diagramme de cas d'utilisation détaillée. Ensuite nous allons définir le diagramme de classe et le diagramme de séquence. La même démarche pour le quatrième chapitre « Gestion de tableaux de bord \& de documents ».
\medskip

Le cinquième chapitre illustre les outils technologiques et matériels utilisés ainsi les diagrammes de composants et déploiement spécifique pour notre application puis les structures et les interfaces développés durant ce stage.
\medskip
Enfin, nous clôturerons notre travail par une conclusion générale et les éventuelles perspectives

\vspace{1cm}



