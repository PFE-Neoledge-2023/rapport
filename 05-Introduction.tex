
\chapter*{Introduction générale}
\addcontentsline{toc}{chapter}{Introduction générale}
\markboth{Introduction générale}{Introduction générale}
\label{chap:introduction}

%\minitoc

Les organismes publics ou privés acquièrent et produisent tout au long de son activité de grandes quantités de documents dans le cadre de leurs activités. Certains sont importants (titres de propriété, contrats, etc.) et doivent être conservés. En outre, les "documents de travail" tels que les procès-verbaux de réunion, les rapports et les documents de bureau peuvent également être utilisés pour la prise de décision. Par conséquent, la gestion et le stockage des documents au sein d'une organisation sont des activités importantes pour gagner du temps et de l'argent, centraliser les documents et protéger les données. Ceci est possible grâce au système de gestion électronique de documents GED. 

\medskip

Pour cela les logiciels de GED sont devenus aujourd'hui la meilleure solution pour effectuer une gestion électronique des documents avec un minimum d'effort.
\medskip

C'est dans ce cadre que se situe notre projet de fin d'études effectuée au sein de la société NeoLedge pendant la période de 06/02/2022 aux 28/05/2022 pour l'obtention du diplôme de licence nationale en technologie d'information.
\medskip

Il s'agit en fait de la conception et du développement d'une application mobile « Cross-Platform » pour offrir aux utilisateurs de l'application « Elise Web » les mêmes fonctionnalités. Cette solution permettra la gestion et la préservation des documents qui sont des activités essentielles. En outre, elle permet de faciliter le travail collaboratif.

\medskip
Le présent rapport synthétise ainsi le déroulement de notre travail sur ce projet. Il est structuré en cinq chapitres :

\medskip
 Le premier chapitre sera consacré au contexte général du projet,  nous présentons en premier lieu, l'organisme d'accueil. Ensuite, nous définissons la problématique à laquelle notre projet répond, ainsi que la solution proposée. Puis nous expliquons les motivations qui ont conduit à la réalisation de ce projet. Enfin, nous décrivons les méthodologies et l'environnement de travail adoptés pour mener à bien ce projet.

\medskip
Dans le deuxième chapitre, "Planification du Backlog Product", nous allons examiner le fonctionnement de l'application à travers  le Product Backlog, en clarifiant les exigences de notre projet en vue d'une bonne étude technique.

\medskip
Les deux derniers chapitres se concentrent sur chaque release de notre projet. Chaque release sera abordée en détaillant les sprints, les phases d'analyse et de conception, jusqu'à la réalisation finale.
\medskip
Enfin, nous clôturerons notre travail par une conclusion générale et les éventuelles perspectives
\vspace{1cm}



