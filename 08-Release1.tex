\chapter*{Release 1}
\addcontentsline{toc}{chapter}{Release 1}
\markboth{Release 1}{Release 1}
\renewcommand\fbox{\fcolorbox{blue}{white}}
\label{chap:release1}
\section*{Introduction}

Ce chapitre présente la première version de notre application. Il est composé de quatre sprints qui ont été réalisés en 8 semaines. Nous avons commencé par la conception de l'application, puis nous avons implémenté les fonctionnalités de base de l'application tels que la gestion des documents et des signatures. Dans ce chapitre, nous décrirons en détail les fonctionnalités de chaque sprint, les défis que nous avons rencontrés et les solutions que nous avons apportées pour les surmonter.

Release 1 : (Du 8 Février Au 19 Avril)

\fbox{\begin{minipage}{30em}
  \textbf{Organisation des sprints :} \\
  Cette release contient les quatre sprints:
  \begin{itemize}
    \item \textbf{Sprint 1:} Préparation de l'environnement du travail et étude de la solution.
    \item \textbf{Sprint 2:} Gestion des signatures.
    \item \textbf{Sprint 3:} Gestion des documents.
    \item \textbf{Sprint 4:} Visualisation et signature de fichiers.
  \end{itemize}
\end{minipage}}

\section{Sprint 1 (Préparation de l'environnement du travail et étude de la solution)}

\subsection{Sprint Goal}
L'objectif de ce sprint est de préparer l'environnement de travail et d'étudier la solution ainsi que les technologies à utiliser afin de pouvoir commencer le développement de l'application, et ensuite, demarrer le développement de l'application elise mobile contenant les fonctionnalités de base (S'authentifier, Se déconnecter).

\subsection{Sprint Backlog}

\begin{adjustwidth}{-1cm}{}
  % \usepackage{longtable}
    
    \begin{longtable}{|c|p{6cm}|c|p{6cm}|c|}
      % \centering
      \hline
      \textbf{ID} & \textbf{User story} & \textbf{ID}  & \textbf{Tâche} & \textbf{Durée} \\
      \hline

    1 & En tant que membre de l'équipe scrum, je
    souhaite comprendre pleinement la fonctionnalité du système Elise afin de reproduire sa logique dans une application mobile. &  1.1 &suivre une formation préparée par la société sur Elise.&3 Jours\\
    \cline{1-5}
    \multirow{4}{*}{2} & \multirow{4}{6cm}{En tant que scrum team, je souhaite me former au développement mobile en utilisant les technologies Ionic Vue et Capacitor. Cette formation doit me permettre de maîtriser les compétences essentielles pour créer des applications mobiles multiplateformes de qualité.}  &  2.1 &Suivre une formation sur Youtube qui explique les notions de base d'Ionic.&\multirow{4}{2cm}{3.5 Jours}\\
    \cline{3-4}
    &  &  2.2 &Suivre une formation sur Youtube qui explique les notions de base du Capacitor.&\\
    \cline{3-4}
    &  &  2.3 &Suivre une formation sur Youtube qui explique les notions de base du Soap.&\\
    \cline{3-4}
    &  &  2.4 &Suivre une formation sur Youtube qui explique les notions de base du .NET core 6.&\\
    \cline{1-5}
    \multirow{7}{*}{3} & \multirow{7}{6cm}{En tant que membre de l'équipe scrum, je souhaite installer et configurer l'environnement de développement.} &  3.1 &Installer VS code.&\multirow{7}{2cm}{0.5 Jours}\\
    \cline{3-4}
    &  &  3.2 &Installer Android Studio.&\\
    \cline{3-4}
    &  &  3.3 &Installer .NET Core 6 .&\\
    \cline{3-4}
    &  &  3.4 &Installer Ionic Version6.&\\
    \cline{3-4}
    &  &  3.5 &Installer Vue Js Version 3.&\\
    \cline{3-4}
    &  &  3.6 &Installer Capacitor Version 4.&\\
    \cline{3-4}
    &  &  3.7 &Installer SoapUi.&\\
    \cline{1-5}
    \multirow{4}{*}{4} & \multirow{4}{6cm}{En tant que développeur, je veux développer des prototypes de l'application mobile afin de tester les fonctionnalités et de valider les choix techniques.} &  4.1 &\textbf{Application 1} Développer une application mobile qui consomme le webservice soap pour afficher les données de la météo.&\multirow{4}{2cm}{4 Jours}\\
    \cline{3-4}
    &  &  4.2 &\textbf{Application 2} Développer une application mobile qui permet la création d'une signature.&\\
    \cline{3-4}
    &  &  4.3 &\textbf{Application 3} Développer une application mobile qui permet de visualiser un fichier PDF.&\\
    \cline{3-4}
    &  &  4.4 &\textbf{Application 4} Développer une application mobile qui permet de signer un fichier PDF en utilisant les deux solutions 2 et 3.&\\
    \cline{1-5}
    % S'authentifier
    \multirow{3}{*}{5} & \multirow{3}{6cm}{En tant que développeur, je veux développer une application mobile qui permet à l'utilisateur de s'authentifier afin d'accéder à l'application.} & 
    
    5.1 &Initialiser un projet Ionic Vue.&\multirow{3}{2cm}{3 Jours}\\
    \cline{3-4}
      & & 5.2 & Préparer les interfaces sur Figma.&\\
    \cline{3-4}
    &  &  5.3 & Créer la page d'authentification. &\\
    \cline{3-4}
    &  &  5.4 & Développer la fonctionnalité d'authentification.&\\
  \hline
  \multirow{3}{*}{6} & En tant qu'utilisateur, je veux déconnecter de l'application Elise Mobile afin de protéger mes informations et données personnelles. La déconnexion doit être facile d'accès et rapide. &  6.1 &Préparer les interfaces sur Figma.&\multirow{3}{2cm}{2 Jours}\\
  \cline{3-4}
  &  &  6.2 &Créer la page de déconnexion.&\\
  \cline{3-4}
  &  &  6.3 &Développer la fonctionnalité de déconnexion.&\\
  \hline
  \caption{Sprint Backlog du Sprint 1}
  \label{tab:sprint-backlog-1}
\end{longtable}
\end{adjustwidth}
\subsection{Implémentation du Sprint 1}
\textbf{•	Diagramme de cas d'utilisation du sprint 1}

% add image
\begin{figure}[H]
  \centering
  \includegraphics[width=0.7\textwidth]{use_case_documents_sprint_1}
  \caption{Diagramme de cas d'utilisation du sprint }
  \label{fig:UseCaseDiagram1}
\end{figure}

\subsubsection{Analyse des besoins:}
\textbf{•	Description textuelle de cas d'utilisation « S'authentifier »}

\begin{longtable}{|p{5cm}|p{10cm}|}
\hline
\textbf{Cas d'utilisation}&S'authentifier (authentification simple)\\
\hline
\textbf{Acteurs}&Utilisateur\\
\hline
\textbf{Pré Condition}&L'utilisateur doit avoir un compte\\
\hline
\textbf{Post Condition}&Authentification\\
\hline
\textbf{Scénario Nominal}&
\vspace{-\baselineskip}
\begin{enumerate}
    \setcounter{enumi}{1}
  \item L'utilisateur saisit l'adresse du serveur
  \item L'utilisateur clique sur le bouton connexion
  \item Le système vérifie si le serveur existe
  \item Le système affiche le formulaire d'authentification
  \item L'utilisateur saisit son identifiant et son mot de passe
  \item L'utilisateur clique sur le bouton connexion
  \item Le système vérifie les cordonnées
  \item Le système redirige l'utilisateur vers la page d'accueil
  \item Le système affiche le message « bienvenue »
\end{enumerate}\\
\hline
\textbf{Scénario Alternatif}&
\vspace{-\baselineskip}
\begin{enumerate}
      \item [4.1] Le serveur n'est pas disponible
      \item [4.2] Le système affiche le message « le serveur n'est pas disponible »
      \item [7.1] Les cordonnées sont incorrectes
      \item [7.2] Le système affiche un message d'erreur
\end{enumerate}\\
\hline
\textbf{Scénario d'exception}&Erreur de connexion\\
\hline
\caption{Description textuelle du diagramme de cas d'utilisation « S'authentifier »}
\label{tab:use_case_simple_authentification}
\end{longtable}

\textbf{•	Description textuelle de cas d'utilisation « Se déconnecter »}

\begin{longtable}{|p{5cm}|p{10cm}|}
\hline
\textbf{Cas d'utilisation}&Se déconnecter\\
\hline
\textbf{Acteurs}&Utilisateur \\
\hline
\textbf{Pré Condition}&L'utilisateur doit être authentifié\\
\hline
\textbf{Post Condition}&Déconnexion\\
\hline
\textbf{Scénario Nominal}&
\vspace{-\baselineskip}
\begin{enumerate}
    \setcounter{enumi}{1}
    \item L'utilisateur clique sur le bouton Déconnecté
    \item Le système redirige l'utilisateur vers la page d'authentification

\end{enumerate}\\
\hline
\caption{Description textuelle du diagramme de cas d'utilisation « Se déconnecter »}
\label{tab:use_case_logout}
\end{longtable}

Pour créer notre plateforme, nous avons pensé qu'il était nécessaire de réaliser une maquette. Les maquettes nous aident à concevoir des interfaces qui répondent aux attentes et aux besoins du client. Elles permettent également de s'assurer que les besoins du client sont adaptés au projet. Nous avons réalisé un prototype de notre système, comme le montrent les figures ci-dessous. Tout au long de notre projet, nous présenterons des maquettes.

\begin{figure}[H]
  \centering
  \includegraphics[width=0.5\textwidth]{design_auth}
  \caption{Maquette de la d'authentification}
  \label{fig:design_auth}
\end{figure}


Pour avoir une représentation temporelle des interactions entre les objets de notre système et de la chronologie des messages échangés entre eux et avec les acteurs nous avons réalisé les diagrammes de séquence représentés ci-dessous

\begin{figure}[H]
  \centering
  \includegraphics[width=0.6\textwidth]{out/diagrams/authentification/simple_login/simple_login}
  \caption{Diagramme de séquence de cas d'utilisation « S'authentifier »}
  \label{fig:sequence_login}
\end{figure}

\begin{figure}[H]
  \centering
  \includegraphics[width=0.4\textwidth]{out/diagrams/authentification/logout/logout}
  \caption{Diagramme de séquence de cas d'utilisation « Se déconnecter »}
  \label{fig:sequence_logout}
\end{figure}


\subsubsection{Analyse détaillée}
La présentation de démarche d'analyse fonctionnelle d'un sprint est très importante pour la satisfaction d'un client parce qu'elle consiste à caractériser les fonctions offertes par un produit.
Donc, nous allons faire l'analyse des différents cas d'utilisation en utilisant le diagramme de classes d'analyse.

Les objets de diagramme d'analyse sont les suivants :
% View
\begin{itemize}
  \item \textbf{View} : représente la vue de l'application.
  \item \textbf{ViewModel} : représente le modèle de la vue.
  \item \textbf{Model} : représente le modèle de l'application.
\end{itemize}

% spacing between paragraphs
\setlength{\parskip}{1em}
% spacing left
\setlength{\parindent}{0em}

\textbf{•	Diagramme de classe d'analyse de sprint 1 }


\begin{figure}[H]
  \centering
  \includegraphics[width=1\textwidth]{dca_sprint1}
  \caption{Diagramme de classe d'analyse de sprint 1}
  \label{fig:class_analyse_auth}
\end{figure}


\subsubsection{Conception}

Après la présentation des diagrammes d'analyse, nous allons présenter dans cette partie les diagrammes de conception.\\ 
\textbf{•	Diagramme de classe de conception de sprint 1}

% add image
\begin{figure}[H]
  \centering
  \includegraphics[width=0.9\textwidth]{dcc_sprint1}
  \caption{Diagramme de classe de conception de sprint 1}
  \label{fig:class_diagram_auth}
\end{figure}


\begin{figure}[H]
  \centering
  \includegraphics[width=0.9\textwidth]{out/diagrams/authentification/sequence_simple_login/sequence_simple_login}
  \caption{Diagramme de séquence de conception de cas d'utilisation « S'authentifier »}
  \label{fig:sequence_conception_login}
\end{figure}

\begin{figure}[H]
  \centering
  \includegraphics[width=1\textwidth]{out/diagrams/authentification/sequence_logout/sequence_logout}
  \caption{Diagramme de séquence de conception de cas d'utilisation « Se déconnecter »}
  \label{fig:sequence_conception_logout}
\end{figure}

\subsubsection{Réalisation}

Après la présentation des diagrammes d'analyse, nous avons présenté dans cette partie des captures d'écran de l'application.

% add image
\begin{figure}[H]
  \centering
  \includegraphics[width=0.7\textwidth, height=0.7\textheight,keepaspectratio=true]{realision_auth}
  \caption{Les interfaces d'authentification}
  \label{fig:realision_auth}
\end{figure}

\subsection{Sprint Review}
Suite à cette Technical Story, nous avons préparé notre environnement de travail où nous aurons les possibilités de traiter les prochains sprints.

\subsection{Sprint Retrospective}

\begin{itemize}
  \item \textbf{Ce qui a bien fonctionné :}
  \begin{itemize}
    \item Nous avons pu suivre les formations préparées par la société.
    \item Nous avons pu installer les outils nécessaires pour le développement.
    \item Nous avons bien mis nos connaissances en pratique dans certains projets préparatoires (Application 1, 2, 3 et 4).
    \item Nous avons pu bien initialiser le projet.
    \item Nous avons pu développer la fonctionnalité d'authentification.
  \end{itemize}
  \item \textbf{Ce qui n'a pas bien fonctionné :}
  
  Nous avons remarqué que le temps de formation est très réduit, nous avons donc décidé de suivre des formations sur Youtube pour nous former sur les technologies à utiliser en plus de la formation préparée par la société.
\end{itemize}

\section{Sprint 2 (Gestion des signatures)}

\subsection{Sprint Goal}

L'objectif de ce sprint est de développer et mettre en place un système de gestion des signatures permettant aux utilisateurs d'ajouter et de visualiser facilement les signatures crée, tout en garantissant la sécurité.

\subsection{Sprint Backlog}


\begin{adjustwidth}{-1cm}{}
  % \usepackage{longtable}
    
    \begin{longtable}{|c|p{6cm}|c|p{6cm}|c|}
      % \centering
      \hline
      \textbf{ID} & \textbf{User story} & \textbf{ID}  & \textbf{Tâche} & \textbf{Durée} \\
      \hline
      \multirow{2}{*}{1} & En tant qu'utilisateur, je veux créer une signature électronique en dessinant ma signature à l'aide de mon doigt ou de mon stylet sur l'écran tactile de mon appareil mobile, afin de la réutiliser facilement lors de la signature des documents. & 1.1 & Préparer les interfaces sur Figma. & \multirow{3}{*}{2.5 Jour} \\
      \cline{3-4}
      & & 1.2 & Développer l'interface de création de signature. & \\
      \cline{3-4}
      & & 1.3 & Développer la fonction qui permet de créer une signature. & \\
      \cline{1-5}
      \multirow{3}{*}{2} & En tant qu'utilisateur, je veux lister les signatures que j'ai créées, afin de les visualiser facilement. & 2.1 & Préparer les interfaces sur Figma. & \multirow{3}{*}{1 Jour} \\
      \cline{3-4}
      & & 2.2 & Développer l'interface de listage des signatures. & \\
      \cline{3-4}
      & & 2.3 & Développer la fonction qui permet de lister les signatures. & \\
      \cline{1-5}
      \multirow{3}{*}{3} & En tant qu'utilisateur, je veux modifier le nom, la description et si elle est par défaut d'une signature que j'ai créée auparavant, afin de la mettre à jour ou de la rendre plus facile à identifier. & 3.1 & Préparer les interfaces sur Figma. & \multirow{3}{*}{0.5 Jour} \\
      \cline{3-4}
      & & 3.2 & Développer l'interface de modification des signatures. & \\
      \cline{3-4}
      & & 3.3 & Développer la fonction qui permet de modifier une signature. & \\
      \cline{1-5}
      \multirow{3}{*}{4} & \multirow{3}{6cm}{En tant qu'utilisateur, je veux supprimer une signature que j'ai créée, afin de ne pas utiliser une signature
      obsolète ou inexacte.} & 4.1 & Préparer les interfaces sur Figma. & \multirow{3}{*}{0.5 Jour} \\
      \cline{3-4}
      & & 4.2 & Développer l'interface de suppression des signatures. & \\
      \cline{3-4}
      & & 4.3 & Développer la fonction qui permet de supprimer une signature. & \\
      \cline{1-5}
      \multirow{2}{*}{5} & \multirow{3}{6cm}{En tant qu'utilisateur, je veux visualiser une signature que j'ai créée, pour m'assurer qu'elle est correcte.} & 5.1 & Préparer les interfaces sur Figma. & \multirow{3}{*}{2 Jour} \\
      \cline{3-4}
      & & 5.2 & Développer l'interface de visualisation des signatures. & \\
      \cline{3-4}
      & & 5.3 & Développer la fonction qui permet de visualiser une signature. & \\
      \cline{1-5}
      \multirow{3}{*}{6} & \multirow{3}{6cm}{En tant que développeur, je veux développer un lecteur de fichier afin de l'utiliser pour la lecture des fichiers d'Elise Mobile.} & 6.1 & Se familiariser avec pdf.js. & \multirow{3}{*}{1 Jour} \\
      \cline{3-4}
      & & 6.2 & Développer le lecteur de fichier. & \\
      \cline{3-4}
      & & 6.3 & Publier le lecteur de fichier sur npm. & \\
  \hline
  \caption{Sprint backlog du Sprint 2}
  \label{tab:sprint-backlog-2}
\end{longtable}
\end{adjustwidth}

\subsection{Implémentation du Sprint 2}
\textbf{•	Diagramme de cas d'utilisation du sprint 2 : « Gestion des signatures »}

% add image
\begin{figure}[H]
  \centering
  \includegraphics[width=0.8\textwidth]{use_case_signature_sprint_2}
  \caption{Diagramme de cas d'utilisation du sprint 2 : « Gestion des signatures »}
  \label{fig:UseCaseDiagram}
\end{figure}

\subsubsection{Analyse des besoins:}
\textbf{•	Description textuelle de cas d'utilisation « Créer une signature »}

\begin{longtable}{|p{5cm}|p{10cm}|}
\hline
\textbf{Cas d'utilisation}&Créer une signature\\
\hline
\textbf{Acteurs}&Utilisateur\\
\hline
\textbf{Pré Condition}&L'utilisateur doit être authentifié\\
\hline
\textbf{Post Condition}&Création d'une signature\\
\hline
\textbf{Scénario Nominal}&
\vspace{-\baselineskip}
\begin{enumerate}
    \setcounter{enumi}{1}
  \item L'utilisateur dessine sa signature sur le pad.
  \item L'utilisateur clique sur le bouton enregistrer.
  \item Le système affiche un panneau d'ajout de nom.
  \item L'utilisateur entre le nom de la signature.
  \item L'utilisateur clique sur le bouton enregistrer.
  \item Le système enregistre la signature.
  \item Le système affiche un message de succès.
  \item Le system cache le panneau d'ajout de nom.
\end{enumerate}\\
\hline
\textbf{Scénario Alternatif}&
\vspace{-\baselineskip}
\begin{enumerate}
      \item [2.1] L'utilisateur ne dessine pas sa signature sur le pad.
      \item [2.2] Le système affiche un message d'erreur pour s'assurer de dessiner la signature.
      \item [5.1] L'utilisateur ne donne pas de nom à la signature.
      \item [5.2] Le système affiche un message d'erreur pour s'assurer de donner un nom à la signature.
\end{enumerate}\\
\hline
\textbf{Scénario d'exception}&Erreur de connexion\\
\hline
\caption{Description textuelle du diagramme de cas d'utilisation « Créer une signature »}
\label{tab:use_case_create_signature}
\end{longtable}

% \textbf{Terminologie paragraphe :} \\
% \textbf{OCR :} signifie Optical Character Recognition (reconnaissance optique de caractères en français). Il s'agit d'un processus de conversion d'images numérisées de textes en fichiers éditables et interprétables par des ordinateurs.
  

\textbf{•	Description textuelle de cas d'utilisation « Afficher les signatures »}

\begin{longtable}{|p{5cm}|p{10cm}|}
\hline
\textbf{Cas d'utilisation}&Afficher les signatures\\
\hline
\textbf{Acteurs}&Utilisateur \\
\hline
\textbf{Pré Condition}&L'utilisateur doit être authentifié\\
\hline
\textbf{Post Condition}&Affichage des signatures\\
\hline
\textbf{Scénario Nominal}&
\vspace{-\baselineskip}
\begin{enumerate}
    \setcounter{enumi}{1}
    \item L'utilisateur clique sur le bouton « Signatures ».
    \item Le système affiche la liste des signatures.
\end{enumerate}\\
\hline
\textbf{Scénario alternatif}&
\begin{enumerate}
  \item [2.1] L'utilisateur n'a pas de signature.
  \item [2.2] Le système affiche un message pour l'inviter à créer une signature.
\end{enumerate}\\
\hline
\textbf{Scénario d'exception}&Erreur de connexion\\
\hline
\caption{Description textuelle du diagramme de cas d'utilisation « Afficher les signatures »}
\label{tab:use_case_view_signature}
\end{longtable}

\textbf{•	Description textuelle de cas d'utilisation « Modifier une signature »}

\begin{longtable}{|p{5cm}|p{10cm}|}
\hline
\textbf{Cas d'utilisation}&Modifier une signature\\
\hline
\textbf{Acteurs}&Utilisateur\\
\hline
\textbf{Pré Condition}&L'utilisateur doit être authentifié et avoir des signatures\\
\hline
\textbf{Post Condition}&Modification d'une signature\\
\hline
\textbf{Scénario Nominal}&
\vspace{-\baselineskip}
\begin{enumerate}
    \setcounter{enumi}{1}
  \item L'utilisateur clique sur le bouton modifier.
  \item Le système affiche un panneau de modification.
  \item L'utilisateur modifie le nom, description ou si la signature est par défaut.
  \item L'utilisateur clique sur le bouton enregistrer.
  \item Le système enregistre la modification.
  \item Le système affiche un message de succès.
  \item Le system cache le panneau de modification.
\end{enumerate}\\
\hline
\textbf{Scénario Alternatif}&
\vspace{-\baselineskip}
\begin{enumerate}
      \item [4.1] L'utilisateur ne remplit pas le champ de nom.
      \item [4.2] Le système affiche un message d'erreur.
\end{enumerate}\\
\hline
\textbf{Scénario d'exception}&Erreur de connexion\\
\hline
\caption{Description textuelle du diagramme de cas d'utilisation « Créer une signature »}
\label{tab:use_case_update_signature}
\end{longtable}


\textbf{•	Description textuelle de cas d'utilisation « Supprimer une signature »}

\begin{longtable}{|p{5cm}|p{10cm}|}
\hline
\textbf{Cas d'utilisation}&Supprimer une signature\\
\hline
\textbf{Acteurs}&Utilisateur \\
\hline
\textbf{Pré Condition}&L'utilisateur doit être authentifié et avoir des signatures.\\
\hline
\textbf{Post Condition}&Suppression  d'une signature\\
\hline
\textbf{Scénario Nominal}&
\vspace{-\baselineskip}
\begin{enumerate}
    \setcounter{enumi}{1}
    \item L'utilisateur clique sur le bouton supprimer.
    \item Le système affiche une alerte de vérification.
    \item L'utilisateur clique sur le bouton confirme.
    \item La signature est supprimer de la base.
    \item Le système affiche un message de succès.
\end{enumerate}\\
\hline
\textbf{Scénario d'exception}&Erreur de connexion\\
\hline
\caption{Description textuelle du diagramme de cas d'utilisation « Supprimer une signature »}
\label{tab:use_case_delete_signature}
\end{longtable}


\textbf{•	Description textuelle de cas d'utilisation « Visualiser une signature »}

\begin{longtable}{|p{5cm}|p{10cm}|}
\hline
\textbf{Cas d'utilisation}&Visualiser une signature\\
\hline
\textbf{Acteurs}&Utilisateur \\
\hline
\textbf{Pré Condition}&L'utilisateur doit être authentifié et avoir des signatures.\\
\hline
\textbf{Post Condition}&Visualisation d'une signature\\
\hline
\textbf{Scénario Nominal}&
\vspace{-\baselineskip}
\begin{enumerate}
    \setcounter{enumi}{1}
    \item L'utilisateur clique sur le bouton visualiser.
    \item Le système affiche la signature.
\end{enumerate}\\
\hline
\textbf{Scénario d'exception}&Erreur de connexion\\
\hline
\caption{Description textuelle du diagramme de cas d'utilisation « Visualiser une signature »}
\label{tab:use_case_view_single_signature}
\end{longtable}

Les maquettes suivantes représentent les interfaces de la page des signatures.

\begin{figure}[H]
  \centering
  \includegraphics[width=0.6\textwidth]{design_signatures1}
  \caption{Maquette de la page des signatures (Lister, créer et supprimer)}
  \label{fig:design_signatures}
\end{figure}

\begin{figure}[H]
  \centering
  \includegraphics[width=0.4\textwidth]{design_signatures2}
  \caption{Maquette de la page des signatures (Visualiser et modifier)}
  \label{fig:design_preview_delete_signature}
\end{figure}

Dans les diagrammes de séquence suivants, nous allons détailler les cas d'utilisation de cette sprint.

\begin{figure}[H]
  \centering
  \includegraphics[width=0.5\textwidth]{out/diagrams/signatures/create/create_signature}
  \caption{Diagramme de séquence de cas d'utilisation « Créer une signature  »}
  \label{fig:sequence_create_signature}
\end{figure}

\begin{figure}[H]
  \centering
  \includegraphics[width=0.5\textwidth]{out/diagrams/signatures/delete/delete_signature}
  \caption{Diagramme de séquence de cas d'utilisation « Supprimer une signature  »}
  \label{fig:sequence_delete_signature}
\end{figure}

\begin{figure}[H]
  \centering
  \includegraphics[width=0.7\textwidth]{out/diagrams/signatures/view/view_signature}
  \caption{Diagramme de séquence de cas d'utilisation « Visualiser une signature  »}
  \label{fig:sequence_view_signature}
\end{figure}

\begin{figure}[H]
  \centering
  \includegraphics[width=0.7\textwidth]{out/diagrams/signatures/update/update_signature}
  \caption{Diagramme de séquence de cas d'utilisation « Modifier une signature  »}
  \label{fig:sequence_update_signature}
\end{figure}

\subsubsection{Analyse détaillée}
Dans cette partie, nous allons illustrer le diagramme de classes participantes représenté ci-dessous.\\

\textbf{•	Diagramme de classe d'analyse de sprint 2 }

\begin{figure}[H]
  \centering
  \includegraphics[width=1\textwidth]{dca_sprint2}
  \caption{Diagramme de classe d'analyse de sprint 2}
  \label{fig:class_analyse_signatures}
\end{figure}


\subsubsection{Conception}
Dans cette partie nous allons représenter le diagramme de classe de conception, ainsi que les diagrammes de séquences de conception de notre deuxième Sprint. \\

\textbf{•	Diagramme de classe de conception de sprint 2 : « Gestion des signatures »}

% add image
\begin{figure}[H]
  \centering
  \includegraphics[width=1\textwidth]{dcc_spint2}
  \caption{Diagramme de classe de conception de sprint 2 : « Gestion des signatures »}
  \label{fig:class_diagram_signatures}
\end{figure}


\begin{figure}[H]
  \centering
  \includegraphics[width=0.8\textwidth]{out/diagrams/signatures/sequence_create/sequence_create_signature}
  \caption{Diagramme de séquence de conception de cas d'utilisation « Créer une signature  »}
  \label{fig:sequence_conception_create_signature}
\end{figure}

\begin{figure}[H]
  \centering
  \includegraphics[width=0.7\textwidth]{out/diagrams/signatures/sequence_view/sequence_view_signature}
  \caption{Diagramme de séquence de conception de cas d'utilisation « Visualiser une signature  »}
  \label{fig:sequence_conception_view_signature}
\end{figure}

\begin{figure}[H]
  \centering
  \includegraphics[width=0.7\textwidth]{out/diagrams/signatures/sequence_delete/sequence_delete_signature}
  \caption{Diagramme de séquence de conception de cas d'utilisation « Supprimer une signature  »}
  \label{fig:sequence_conception_delete_signature}
\end{figure}

\begin{figure}[H]
  \centering
  \includegraphics[width=0.7\textwidth]{out/diagrams/signatures/sequence_update/sequence_update_signature}
  \caption{Diagramme de séquence de conception de cas d'utilisation « Modifier une signature  »}
  \label{fig:sequence_conception_update_signature}
\end{figure}

\subsubsection{Réalisation}

Après la présentation des diagrammes d'analyse, nous allons présenter dans cette partie des captures d'écran qui représentent les interfaces de gestion des signatures.
\begin{figure}[H]
  \centering
  \includegraphics[width=0.7\textwidth]{realisation_signatures1}
  \caption{Captures de la page des signatures (Lister, créer et supprimer)}
  \label{fig:realisation_signatures}
\end{figure}

\begin{figure}[H]
  \centering
  \includegraphics[width=0.5\textwidth]{realisation_signatures2}
  \caption{Captures de la page des signatures (Visualiser et supprimer)}
  \label{fig:realisation_preview_delete_signature}
\end{figure}


\subsection{Développement de lecteur de fichier}
\subsubsection{Besoins et spécifications}
Ce lecteur est conçu pour permettre la visualisation de fichiers PDF dans un environnement Vue.js. Les principaux besoins et spécifications du package étaient les suivants:
\begin{itemize}
  \item Chargement et affichage de fichiers PDF locaux sous forme base64.
  \item Prise en charge de la navigation par page ou l'affichage de tous les pages dans le document PDF.
  \item Gestion des événements liés à l'interaction de l'utilisateur avec le visualiseur PDF.
\end{itemize}

\subsubsection{Choix techniques}
Pour réaliser ce lecteur, les choix techniques suivants ont été faits :
\begin{itemize}
  \item Utilisation de la bibliothèque PDF.js, une bibliothèque JavaScript populaire et open-source, pour le rendu et l'interprétation des fichiers PDF.
  \item Utilisation de Vue.js pour la création de composants réutilisables et la gestion des événements liés à l'interaction de l'utilisateur avec le visualiseur PDF.
\end{itemize}

\subsubsection{Réalisation}

\textbf{•	Architecture du package}\\
L'architecture du lecteur est basée sur une combinaison de la bibliothèque PDF.js et du framework Vue.js. Le package comprend plusieurs modules qui gèrent le chargement, le rendu et l'interaction avec les fichiers PDF.

\textbf{Développement des fonctionnalités }\\
Les fonctionnalités développées comprennent :

\begin{itemize}
  \item Le chargement et l'affichage des fichiers PDF à l'aide de la bibliothèque PDF.js.
  \item Les fonctionnalités d'interaction avec les fichiers PDF, la navigation par page et l'affichage de tous les pages dans le document PDF.
\end{itemize}

\textbf{Publication du package}\\
Nous avons choisi de nommer le package neo-pdf-viewer. Il est disponible sur le lien suivant : \url{https://www.npmjs.com/package/neo-pdf-viewer}

\begin{figure}
  \centering
  \includegraphics[width=0.7\textwidth]{neo_pdf_npm}
  \caption{Publication du package neo-pdf-viewer}
  \label{fig:npm_package}
\end{figure}

\subsection{Performances}
Le package a été optimisé pour offrir des performances satisfaisantes en termes de chargement et de rendu des fichiers PDF.
En moyenne, pour un fichier de 1-5 pages avec de détails moyens, le temps de chargement est de 0.35 à 1 seconde.

\begin{figure}[H]
  \centering
  \includegraphics[width=0.7\textwidth]{rendering_pdf_viewer}
  \caption{Performances du package neo-pdf-viewer}
  \label{fig:performance_package}
\end{figure}


\subsection{Sprint Review:}
A la fin de ce sprint, nous avons planifié une réunion dans la société Neoledge afin de vérifier notre démarche de travail par rapport au besoin de client tout en respectant le délai que nous avons prévu.

Nous avons fait une démonstration durant laquelle nous allons présenter notre incrément :

\begin{itemize}
  \item La création d'une signature.
  \item La visualisation de la signature.
  \item La modification de la signature.
  \item La suppression de la signature.
  \item La création et la publication du package neo-pdf-viewer.
\end{itemize}

\subsection{Sprint retrospective:}

Après la Sprint Review, nous avons réfléchi à des pistes pour améliorer la qualité et l'efficacité de notre application.

\noindent\textbf{•	Ce qui s'est bien passé :}
\begin{itemize}
  \item Nous avons bien partagé les tâches entre nous à travers le logiciel Azure DevOps. 
  \item Nous avons terminé le sprint dans le délai.
\end{itemize}

\noindent\textbf{•	Ce qui s'est mal passé :}
\begin{itemize}
  \item L'intégration du bibliothèque vue-signature-pad
  \item Un problème rencontré lors de la sauvegarde de la signature était que l'image enregistrée occupait la totalité de l'espace du pad, ce qui nécessitait ensuite une étape de découpe pour obtenir uniquement la partie signée.
  \item Le démarrage de neo-pdf-viewer en tant que projet open-source était un peu difficile.
\end{itemize}

% sprint =3
\section{Sprint 3 « Gestion des documents »}
\subsection{Sprint Goal:}

L'objectif de ce sprint est de développer et mettre en place un système de gestion des documents permettant aux utilisateurs de consulter les informations du document, consulter et gérer un fichier dans un document, consulter et gérer les tâches dans un document, chercher des documents et consulter la liste des documents en favoris

\pagebreak

\subsection{Sprint Backlog « Gestion des documents »}

\begin{adjustwidth}{-1cm}{}
  % \usepackage{longtable}
    
    \begin{longtable}{|c|p{6cm}|c|p{6cm}|c|}
      % \centering
      \hline
      \textbf{ID} & \textbf{User story} & \textbf{ID}  & \textbf{Tâche} & \textbf{Durée} \\
      \hline
      % Chercher des documents
      \multirow{3}{*}{1} & \multirow{3}{6cm}{En tant qu'utilisateur, je veux rechercher des documents en utilisant différents critères, afin de trouver rapidement les documents dont j'ai besoin.} & 1.1 & Préparer les interfaces sur Figma. & \multirow{3}{*}{2 Jour} \\
      \cline{3-4}
      & & 1.2 & Développer l'interface de recherche des documents. & \\
      \cline{3-4}
      & & 1.3 & Développer une fonction qui permet de chercher des documents. & \\
      \cline{1-5}
      % Recherche a l'aide de code QR
      2 & En tant qu'utilisateur, je veux rechercher un document en scannant son code QR, afin de le trouver rapidement. & 2.1 & Développer une fonction qui permet de rechercher un document à l'aide d'un code QR. & 2 Jour \\
      \cline{1-5}
      % Consulter la liste des documents en favoris
      \multirow{3}{*}{3} & \multirow{3}{6cm}{En tant qu'utilisateur, je veux afficher mes documents favoris, afin de trouver rapidement les documents dont j'ai besoin.}  & 3.1 & Préparer les interfaces sur Figma. & \multirow{3}{*}{2 Jour} \\
      \cline{3-4}
      & & 3.2 & Développer l'interface des favoris. & \\
      \cline{3-4}
      & & 3.3 & Développer une fonction qui permet de récupérer la liste des documents en favoris. & \\
      \cline{1-5}
      % Accéder à un document
      \multirow{4}{*}{4} & En tant qu'utilisateur, je veux récupérer les informations relatives à un document, telles que la date de création et les tâches associées, afin de mieux comprendre son contexte, les modifier ou de les signer. & 4.1 & Préparer les interfaces sur Figma. & \multirow{3}{*}{2 Jour} \\
      \cline{3-4}
      & & 4.2 & Développer l'interface des documents. & \\
      \cline{3-4}
      & & 4.3 & Développer une fonction qui permet d'accéder à un document. & \\
      \cline{1-5}
      % Télécharger les fichiers d'un document
      \multirow{2}{*}{5} & \multirow{2}{6cm}{En tant qu'utilisateur, je veux télécharger les fichiers attachés à un document, afin de garder une copie locale.} & 5.1 & Développer l'interface de téléchargement des fichiers. & \multirow{2}{*}{2 Jour} \\
      \cline{3-4}
      & & 5.2 & Développer une fonction qui permet de télécharger les fichiers d'un document. & \\
      \cline{1-5}
      % Consulter la liste des fichiers d'un document
      \multirow{3}{*}{6} & \multirow{3}{6cm}{En tant qu'utilisateur, je veux consulter la liste des fichiers attachés à un document, afin de les visualiser.} & 6.1 & Préparer les interfaces sur Figma. & \multirow{3}{*}{2 Jour} \\
      \cline{3-4}
      & & 6.2 & Développer l'interface de liste des fichiers. & \\
      \cline{3-4}
      & & 6.3 & Développer une fonction qui permet de récupérer la liste des fichiers d'un document. & \\
      \cline{1-5}
      % Ajouter un fichier à un document
      \multirow{3}{*}{7} & En tant qu'utilisateur, je veux ajouter un fichier à un document, afin de le partager avec les autres utilisateurs. & 7.1 & Développer l'interface d'ajout des fichiers. & \multirow{3}{*}{2 Jour} \\
      \cline{3-4}
      & & 7.2 & Développer une fonction qui permet ajouter un fichier à un document. & \\
      \cline{1-5}
      % Scanner un fichier
      \multirow{3}{*}{8} & \multirow{3}{6cm}{En tant qu'utilisateur, je veux scanner un fichier à l'aide de la caméra de mon appareil, afin de le joindre à un document.} & 8.1 & Préparer les interfaces sur Figma. & \multirow{3}{*}{2 Jour} \\
      \cline{3-4}
      & & 8.2 & Développer l'interface de scan des fichiers. & \\
      \cline{3-4}
      & & 8.3 & Développer une fonction qui permet de scanner un fichier. & \\
      \cline{1-5}
      % Consulter la liste des tâches d'un document
      \multirow{3}{*}{9} & \multirow{3}{6cm}{En tant qu'utilisateur, je veux consulter la liste des tâches associées à un document, afin de suivre l'état d'avancement des tâches et de savoir qui est responsable de chaque tâche.} & 9.1 & Préparer les interfaces sur Figma. & \multirow{3}{*}{2 Jour} \\
      \cline{3-4}
      & & 9.2 & Développer l'interface de liste des tâches. & \\
      \cline{3-4}
      & & 9.3 & Développer une fonction qui permet de récupérer la liste des tâches d'un document. & \\
      \cline{1-5}
      % Ajouter une tâche à un document
      \multirow{3}{*}{10} & \multirow{3}{6cm}{En tant qu'utilisateur, je veux demander une tâche dans un document en assignant une personne responsable, afin de faciliter la collaboration et le suivi des tâches.} & 10.1 & Préparer l'interface sur Figma. & \multirow{3}{*}{2 Jour} \\
      \cline{3-4}
      & & 10.2 & Développer l'interface d'ajout des tâches. & \\
      \cline{3-4}
      & & 10.3 & Développer une fonction qui permet d'ajouter une tâche à un document. & \\
      \cline{1-5}
      % Transférer une tâche 
      \multirow{3}{*}{11} & \multirow{3}{6cm}{En tant qu'utilisateur, je veux transferer une tâche dans un document a un autre utilisateur, afin que ce dernier puisse la prendre en charge.} & 11.1 & Préparer l'interface sur Figma. & \multirow{3}{*}{2 Jour} \\
      \cline{3-4}
      & & 11.2 & Développer l'interface de transfert des tâches. & \\
      \cline{3-4}
      & & 11.3 & Développer une fonction qui permet de transférer une tâche dans un document à un autre utilisateur. & \\
      \cline{1-5}
      % traitement d'une tâche
      \multirow{3}{*}{12} &\multirow{3}{6cm}{En tant qu'utilisateur, je veux terminer une tâche dans un document, afin de clôturer la tâche et de faciliter le suivi du projet.} & 12.1 & Préparer les interfaces sur Figma. & \multirow{3}{*}{2 Jour} \\
      \cline{3-4}
      & & 12.2 & Développer les interfaces de traitement des tâches. & \\
      \cline{3-4}
      & & 12.3 & Développer une fonction qui permet de traiter une tâche dans un document. & \\
      \cline{1-5}
  \hline
  \caption{Sprint backlog du Sprint 3}
  \label{tab:sprint-backlog-3}
\end{longtable}
\end{adjustwidth}



\subsection{Implémentation du Sprint 3}
\textbf{•	Diagramme de cas d'utilisation du sprint 3 : « Gestion des documents »}

% add image
\begin{figure}[H]
  \centering
  \includegraphics[width=0.8\textwidth]{use_case_documents_sprint_3}
  \caption{Diagramme de cas d'utilisation du sprint 3 : « Gestion des documents »}
  \label{fig:UseCaseDiagramSprint3}
\end{figure}


\subsubsection{Analyse des besoins:}
\textbf{•	Description textuelle de cas d'utilisation « Accéder à un document »}

\begin{longtable}{|p{5cm}|p{10cm}|}
\hline
\textbf{Cas d'utilisation}&Accéder à un document\\
\hline
\textbf{Acteurs}&Utilisateur\\
\hline
\textbf{Pré Condition}&Le document existe\\
\hline
\textbf{Post Condition}&Consultation d'un document\\
\hline
\textbf{Scénario Nominal}&
\vspace{-\baselineskip}
\begin{enumerate}
    \setcounter{enumi}{1}
    \item L'utilisateur clique sur le document.
    \item Le système affiche l'interface de document.
    
\end{enumerate}\\
\hline
\textbf{Scénario Alternatif}&
\vspace{-\baselineskip}
\begin{enumerate}
    \setcounter{enumi}{2}
    \item Aucun résultat.
\end{enumerate}\\
\hline
\textbf{Scénario d'exception}&Erreur de connexion\\
\hline
\caption{Description textuelle de cas d'utilisation « Accéder à un document »}
\label{tab:DescriptionTextuelleDeCasDUtilisationAccéderAUnDocument}
\end{longtable}

\textbf{•	Description textuelle de cas d'utilisation « Télécharger un document »}

\begin{longtable}{|p{5cm}|p{10cm}|}
\hline
\textbf{Cas d'utilisation}& Télécharger un document\\
\hline
\textbf{Acteurs}&Utilisateur\\
\hline
\textbf{Pré Condition}&Le document existe\\
\hline
\textbf{Post Condition}&Téléchargement d'un document\\
\hline
\textbf{Scénario Nominal}&
\vspace{-\baselineskip}
\begin{enumerate}
    \setcounter{enumi}{1}
    \item L'utilisateur clique sur le bouton des actions.
    \item Le système affiche le menu des actions.
    \item L'utilisateur clique sur le bouton de téléchargement.
    \item Le système télécharge tout les fichiers du document.
    \item Le système affiche un message de succès.
\end{enumerate}\\
\hline
\textbf{Scénario Alternatif}&
\vspace{-\baselineskip}
\begin{enumerate}
  \item [4.1] Le système n'a pas le droit de télécharger les fichiers.
  \item [4.2] Le système affiche un message d'erreur.
\end{enumerate}\\
\hline
\textbf{Scénario d'exception}&Erreur de connexion\\
\hline
\caption{Description textuelle de cas d'utilisation « Télécharger un document »}
\label{tab:DescriptionTextuelleDeCasDUtilisationTéléchargerUnDocument}
\end{longtable}


\textbf{•	Description textuelle de cas d'utilisation « consulter la liste des fichiers d'un document »}

\begin{longtable}{|p{5cm}|p{10cm}|}
\hline
\textbf{Cas d'utilisation}&Consulter la liste des fichiers d'un document\\
\hline
\textbf{Acteurs}&Utilisateur\\
\hline
\textbf{Pré Condition}&Le document existe\\
\hline
\textbf{Post Condition}&Consultation de la liste des fichiers\\
\hline
\textbf{Scénario Nominal}&
\vspace{-\baselineskip}
\begin{enumerate}
    \setcounter{enumi}{1}
    \item L'utilisateur clique sur le bouton fichier.
    \item Le système affiche l'interface de la liste des fichiers.
    
\end{enumerate}\\
\hline
\textbf{Scénario Alternatif}&
\vspace{-\baselineskip}
\begin{enumerate}
    \setcounter{enumi}{2}
    \item Aucun résultat.
\end{enumerate}\\
\hline
\textbf{Scénario d'exception}&Erreur de connexion\\
\hline
\caption{Description textuelle de cas d'utilisation « consulter la liste des fichiers d'un document »}
\label{tab:DescriptionTextuelleDeCasDUtilisationConsulterLaListeDesFichiersDUnDocument}
\end{longtable}

\textbf{•	Description textuelle de cas d'utilisation « ajouter un fichier à un document »}

\begin{longtable}{|p{5cm}|p{10cm}|}
\hline
\textbf{Cas d'utilisation}&Ajouter un fichier à un document\\
\hline
\textbf{Acteurs}&Utilisateur\\
\hline
\textbf{Pré Condition}&Le document existe\\
\hline
\textbf{Post Condition}&Fichier ajouté\\
\hline
\textbf{Scénario Nominal}&
\vspace{-\baselineskip}
\begin{enumerate}
    \setcounter{enumi}{1}
    \item L'utilisateur clique sur le bouton ajouter.
    \item Le système affiche l'interface d'ajout des fichiers.
    \item L'utilisateur ajoute le fichier et clique sur le bouton confirmer.
    \item le système affiche un message des succès.
    
    
\end{enumerate}\\
\hline
\textbf{Scénario Alternatif}&
\vspace{-\baselineskip}
\begin{enumerate}
    \setcounter{enumi}{3}
    \item L'utilisateur annule l'ajout du fichier.
\end{enumerate}\\
\hline
\textbf{Scénario d'exception}&Erreur de connexion\\
\hline
\caption{Description textuelle de cas d'utilisation « ajouter un fichier à un document »}
\label{tab:DescriptionTextuelleDeCasDUtilisationAjouterUnFichierAUnDocument}

\end{longtable}

\textbf{•	Description textuelle de cas d'utilisation « Scanner un fichier »}

\begin{longtable}{|p{5cm}|p{10cm}|}
\hline
\textbf{Cas d'utilisation}&Scanner un fichier\\
\hline
\textbf{Acteurs}&Utilisateur\\
\hline
\textbf{Pré Condition}&Le document existe\\
\hline
\textbf{Post Condition}&Fichier ajouté\\
\hline
\textbf{Scénario Nominal}&
\vspace{-\baselineskip}
\begin{enumerate}
    \setcounter{enumi}{1}
    \item L'utilisateur clique sur le bouton "liste des actions".
    \item Le système affiche la liste des actions.
    \item L'utilisateur clique sur le bouton "scanner un fichier".
    \item Le système affiche la page de scan.
    \item L'utilisateur clique sur le bouton "ajouter"
    \item Le système affiche la caméra de l'appareil.
    \item L'utilisateur prend une photo.
    \item Le système affiche la photo prise.
    \item L'utilisateur clique sur le bouton "valider".
    \item Le système scanne la photo.
    \item Le système affiche la photo avec un cadre jaune autour du fichier.
    \item L'utilisateur recadre le'image
    \item L'utilisateur valide le scan.
    \item Le système affiche la liste des images scannées.
    \item L'utilisateur peut supprimer une image.
    \item L'utilisateur clique sur le bouton "suivant".
    \item Le système affiche la liste des images scannés sous forme de vignettes.
    \item L'utilisateur peut changer l'ordre des images.
    \item L'utilisateur clique sur le bouton "valider".
    \item Le système affiche le panneau de création de fichier.
    \item L'utilisateur clique sur le bouton "valider".
    \item Le système enregistre les images scannées.
    \item Le système affiche un message de succès.
\end{enumerate}\\
\hline
\textbf{Scénario Alternatif}&
\vspace{-\baselineskip}
\begin{enumerate}
    \setcounter{enumi}{3}
    \item L'utilisateur annule l'ajout du fichier.
\end{enumerate}\\
\hline
\textbf{Scénario d'exception}&Erreur de connexion\\
\hline
\caption{Description textuelle de cas d'utilisation « Scanner un fichier »}
\label{tab:DescriptionTextuelleDeCasDUtilisationScannerUnFichier}

\end{longtable}


% \textbf{•	Description textuelle de cas d'utilisation « supprimer un fichier dans un document »}

% \begin{longtable}{|p{5cm}|p{10cm}|}
% \hline
% \textbf{Cas d'utilisation}&Supprimer un fichier d'un document\\
% \hline
% \textbf{Acteurs}&Utilisateur\\
% \hline
% \textbf{Pré Condition}&Le document existe\\
% \hline
% \textbf{Post Condition}&Fichier supprimé\\
% \hline
% \textbf{Scénario Nominal}&
% \vspace{-\baselineskip}
% \begin{enumerate}
%     \setcounter{enumi}{1}
%     \item L'utilisateur clique sur le bouton supprimer.
%     \item Le système affiche une alerte de confirmation.
%     \item L'utilisateur clique sur le bouton confirmer.
%     \item le système supprime le fichier et affiche un message des succès.
    
    
    
% \end{enumerate}\\
% \hline
% \textbf{Scénario Alternatif}&
% \vspace{-\baselineskip}
% \begin{enumerate}
%     \setcounter{enumi}{3}
%     \item L'utilisateur annule la suppression du fichier.
% \end{enumerate}\\
% \hline
% \textbf{Scénario d'exception}&Erreur de connexion\\
% \hline
% \end{longtable}



\textbf{•	Description textuelle de cas d'utilisation « consulter la liste des tâches d'un document »}

\begin{longtable}{|p{5cm}|p{10cm}|}
\hline
\textbf{Cas d'utilisation}&Consulter la liste des tâches d'un document\\
\hline
\textbf{Acteurs}&Utilisateur\\
\hline
\textbf{Pré Condition}&Le document existe\\
\hline
\textbf{Post Condition}&Consultation des tâches\\
\hline
\textbf{Scénario Nominal}&
\vspace{-\baselineskip}
\begin{enumerate}
    \setcounter{enumi}{1}
    \item L'utilisateur clique sur le bouton tâches.
    \item Le système affiche l'interface de la liste des tâches.
    
\end{enumerate}\\
\hline
\textbf{Scénario Alternatif}&
\vspace{-\baselineskip}
\begin{enumerate}
    \setcounter{enumi}{3}
    \item Aucun tâche trouvée.
\end{enumerate}\\
\hline
\textbf{Scénario d'exception}&Erreur de connexion\\
\hline
\caption{Description textuelle de cas d'utilisation « consulter la liste des tâches d'un document »}
\label{tab:DescriptionTextuelleDeCasDUtilisationConsulterLaListeDesTachesDUnDocument}
\end{longtable}



\textbf{•	Description textuelle de cas d'utilisation « Demander une tâche »}

\begin{longtable}{|p{5cm}|p{10cm}|}
\hline
\textbf{Cas d'utilisation}&Demander une tâche\\
\hline
\textbf{Acteurs}&Utilisateur\\
\hline
\textbf{Pré Condition}&La document a au moins une tâche\\
\hline
\textbf{Post Condition}&Tâche demandée\\
\hline
\textbf{Scénario Nominal}&
\vspace{-\baselineskip}
\begin{enumerate}
    \setcounter{enumi}{1}
    \item L'utilisateur clique sur le bouton demander.
    \item Le système affiche un modal de demande de tâche.
    \item L'utilistauer rempli le formulaire.
    \item L'utilisateur clique sur le bouton confirmer.
    \item Le système cache le modal de demande de tâche.
    \item Le système affiche un message de succès.
\end{enumerate}\\
\hline
\textbf{Scénario Alternatif}&
\vspace{-\baselineskip}
\begin{enumerate}
    \setcounter{enumi}{4}
    \item L'utilisateur annule la demande de tâche.
    \item Le système cache le modal de demande de tâche.
\end{enumerate}\\
\hline
\textbf{Scénario d'exception}&Erreur de connexion\\
\hline
\caption{Description textuelle de cas d'utilisation « Demander une tâche »}
\label{tab:DescriptionTextuelleDeCasDUtilisationDemanderUneTache}
\end{longtable}


\textbf{•	Description textuelle de cas d'utilisation « Tranférer une tâche »}

\begin{longtable}{|p{5cm}|p{10cm}|}
\hline
\textbf{Cas d'utilisation}&Tranférer une tâche\\
\hline
\textbf{Acteurs}&Utilisateur\\
\hline
\textbf{Pré Condition}&La document a au moins une tâche\\
\hline
\textbf{Post Condition}&Tâche transférée\\
\hline
\textbf{Scénario Nominal}&
\vspace{-\baselineskip}
\begin{enumerate}
    \setcounter{enumi}{1}
    \item L'utilisateur clique sur le bouton transférer.
    \item Le système affiche un modal de transfert de tâche.
    \item L'utilistauer rempli le formulaire.
    \item L'utilisateur clique sur le bouton confirmer.
    \item Le système cache le modal de transfert de tâche.
    \item Le système affiche un message de succès.
\end{enumerate}\\
\hline
\textbf{Scénario Alternatif}&
\vspace{-\baselineskip}
\begin{enumerate}
    \setcounter{enumi}{4}
    \item L'utilisateur annule la transfert de tâche.
    \item Le système cache le modal de transfert de tâche.
\end{enumerate}\\
\hline
\textbf{Scénario d'exception}&Erreur de connexion\\
\hline
\caption{Description textuelle de cas d'utilisation « Tranférer une tâche »}
\label{tab:DescriptionTextuelleDeCasDUtilisationTranfererUneTache}
\end{longtable}

\textbf{•	Description textuelle de cas d'utilisation « Traiter une tâche »}

\begin{longtable}{|p{5cm}|p{10cm}|}
\hline
\textbf{Cas d'utilisation}&Traiter une tâche\\
\hline
\textbf{Acteurs}&Utilisateur\\
\hline
\textbf{Pré Condition}&La document a au moins une tâche\\
\hline
\textbf{Post Condition}&Tâche terminée\\
\hline
\textbf{Scénario Nominal}&
\vspace{-\baselineskip}
\begin{enumerate}
    \setcounter{enumi}{1}
    \item L'utilisateur clique sur le bouton traiter.
    \item Le système affiche un modal de confirmation de traitement de tâche.
    \item L'utilisateur clique sur le bouton confirmer.
    \item Le système cache le modal de confirmation de traitement de tâche.
    \item Le système affiche un message de succès.
\end{enumerate}\\
\hline
\textbf{Scénario Alternatif}&
\vspace{-\baselineskip}
\begin{enumerate}
    \setcounter{enumi}{4}
    \item L'utilisateur annule la traitement de tâche.
    \item Le système cache le modal de confirmation de traitement de tâche.
\end{enumerate}\\
\hline
\textbf{Scénario d'exception}&Erreur de connexion\\
\hline
\caption{Description textuelle de cas d'utilisation « Traiter une tâche »}
\label{tab:DescriptionTextuelleDeCasDUtilisationTraiterUneTache}
\end{longtable}



\textbf{•	Description textuelle de cas d'utilisation « Consulter la liste des documents en favoris  »}

\begin{longtable}{|p{5cm}|p{10cm}|}
\hline
\textbf{Cas d'utilisation}&Consulter la liste des documents en favoris\\
\hline
\textbf{Acteurs}&Utilisateur\\
\hline
\textbf{Pré Condition}&L'utilisateur a au moins un document en favoris\\
\hline
\textbf{Post Condition}&Affichage de la liste des documents en favoris\\
\hline
\textbf{Scénario Nominal}&
\vspace{-\baselineskip}
\begin{enumerate}
    \setcounter{enumi}{1}
    \item L'utilisateur clique sur le bouton favoris.
    \item Le système affiche la liste des documents en favoris.
\end{enumerate}\\
\hline
\textbf{Scénario Alternatif}&
\vspace{-\baselineskip}
\begin{enumerate}
    \setcounter{enumi}{2}
    \item Aucun document en favoris.
\end{enumerate}\\
\hline
\textbf{Scénario d'exception}&Erreur de connexion\\
\hline
\caption{Description textuelle de cas d'utilisation « Consulter la liste des documents en favoris  »}
\label{tab:DescriptionTextuelleDeCasDUtilisationConsulterLaListeDesDocumentsEnFavoris}
\end{longtable}




\textbf{•	Description textuelle de cas d'utilisation « Chercher des documents »}

\begin{longtable}{|p{5cm}|p{10cm}|}
\hline
\textbf{Cas d'utilisation}&Chercher des documents\\
\hline
\textbf{Acteurs}&Utilisateur\\
\hline
\textbf{Pré Condition}&\\
\hline
\textbf{Post Condition}&Liste des documents correspondant à la recherche\\
\hline
\textbf{Scénario Nominal}&
\vspace{-\baselineskip}
\begin{enumerate}
    \setcounter{enumi}{1}
    \item L'utilisateur clique sur le bouton rechercher.
    \item Le système affiche l'inteface de recherche des documents.
    \item L'utilisateur rempli le formulaire de recherche.
    \item Le système affiche la liste des documents correspondant à la recherche.
\end{enumerate}\\
\hline
\textbf{Scénario Alternatif}&
\vspace{-\baselineskip}
\begin{enumerate}
    \setcounter{enumi}{4}
    \item Aucun document correspondant à la recherche.
\end{enumerate}\\
\hline
\textbf{Scénario d'exception}&Erreur de connexion\\
\hline
\caption{Description textuelle de cas d'utilisation « Chercher des documents »}
\label{tab:DescriptionTextuelleDeCasDUtilisationChercherDesDocuments}
\end{longtable}

\textbf{•	Description textuelle de cas d'utilisation « Recherche a l'aide de code QR »}

\begin{longtable}{|p{5cm}|p{10cm}|}
\hline
\textbf{Cas d'utilisation}&Recherche a l'aide de code QR\\
\hline
\textbf{Acteurs}&Utilisateur\\
\hline
\textbf{Pré Condition}&\\
\hline
\textbf{Post Condition}&Document affiché\\
\hline
\textbf{Scénario Nominal}&
\vspace{-\baselineskip}
\begin{enumerate}
    \setcounter{enumi}{1}
    \item L'utilisateur clique sur le bouton scanner.
    \item Le système affiche l'inteface de scan de code QR.
    \item L'utilisateur scan le code QR.
    \item Le système recherche le document correspondant au code QR.
    \item Le système affiche le document.
\end{enumerate}\\
\hline
\textbf{Scénario Alternatif}&
\vspace{-\baselineskip}
\begin{enumerate}
    \setcounter{enumi}{5}
    \item Aucun document correspondant au code QR.
\end{enumerate}\\
\hline
\textbf{Scénario d'exception}&Erreur de connexion\\
\hline
\caption{Description textuelle de cas d'utilisation « Chercher des documents »}
\label{tab:DescriptionTextuelleDeCasDUtilisationChercherDesDocumentsQR}
\end{longtable}


Les maquettes suivantes représentent les interfaces de chaque cas d'utilisation de cette sprint. \\

\begin{figure}[H]
  \centering
  \includegraphics[width=0.6\textwidth]{design_show_documents}
  \caption{Maquette de l'affichage des documents}
  \label{fig:design_show_documents}
\end{figure}

\begin{figure}[H]
  \centering
  \includegraphics[width=0.6\textwidth]{design_content}
  \caption{Maquette de l'affichage de contenu d'un document}
  \label{fig:design_show_document}
\end{figure}

\begin{figure}[H]
  \centering
  \includegraphics[width=0.6\textwidth]{design_task_managment}
  \caption{Maquette de l'affichage de gestion des tâches}
  \label{fig:design_manage_tasks}
\end{figure}

Dans les diagrammes de séquence suivants, nous allons détailler les cas d'utilisation de cette sprint.\\

% Add sequence system
\begin{figure}[H]
  \centering
  \includegraphics[width=0.5\textwidth]{out/diagrams/documents/preview/preview_document}
  \caption{Diagramme de séquence de cas d'utilisation « Accéder à un document  »}
  \label{fig:sequence_Accederaundocument}
\end{figure}
\begin{figure}[H]
  \centering
  \includegraphics[width=0.5\textwidth]{out/diagrams/documents/download/download}
  \caption{Diagramme de séquence de cas d'utilisation « Télécharger un document  »}
  \label{fig:sequence_telechargerundocument}
\end{figure}
\begin{figure}[H]
  \centering
  \includegraphics[width=0.5\textwidth]{out/diagrams/documents/previewFiles/preview_files_document}
  \caption{Diagramme de séquence de cas d'utilisation « Consulter la liste des fichiers d'un document  »}
  \label{fig:sequence_previewFiles}
\end{figure}
\begin{figure}[H]
  \centering
  \includegraphics[width=0.5\textwidth]{out/diagrams/documents/previewTasks/preview_tasks_document}
  \caption{Diagramme de séquence de cas d'utilisation « Consulter la liste des tâches d'un document  »}
  \label{fig:sequence_previewTasks}
\end{figure}
\begin{figure}[H]
  \centering
  \includegraphics[width=0.4\textwidth]{out/diagrams/documents/transfer_task/transfer_task}
  \caption{Diagramme de séquence de cas d'utilisation « Transferer une tâche  »}
  \label{fig:sequence_transfer_task}
\end{figure}
\begin{figure}[H]
  \centering
  \includegraphics[width=0.4\textwidth]{out/diagrams/documents/traiter_task/traiter_task}
  \caption{Diagramme de séquence de cas d'utilisation « Traiter une tâche  »}
  \label{fig:sequence_traiter_task}
\end{figure}
\begin{figure}[H]
  \centering
  \includegraphics[width=0.5\textwidth]{out/diagrams/documents/add_file/add_file}
  \caption{Diagramme de séquence de cas d'utilisation « Ajoutez un fichier à un document  »}
  \label{fig:sequence_add_file}
\end{figure}
\begin{figure}[H]
  \centering
  \includegraphics[height=1\textheight]{out/diagrams/documents/add_scan_file/add_scan_file}
  \caption{Diagramme de séquence de cas d'utilisation « Ajoutez un fichier à un document via scan  »}
  \label{fig:sequence_add_scan_file}
\end{figure}
\begin{figure}[H]
  \centering
  \includegraphics[width=0.7\textwidth]{out/diagrams/documents/favoris/favorit_document}
  \caption{Diagramme de séquence de cas d'utilisation « Consulter la liste des documents en favoris  »}
  \label{fig:sequence_favorit_document}
\end{figure}
\begin{figure}[H]
  \centering
  \includegraphics[width=0.7\textwidth]{out/diagrams/documents/chercher/charcher_document}
  \caption{Diagramme de séquence de cas d'utilisation « Chercher des documents  »}
  \label{fig:sequence_charcher_document}
\end{figure}
\begin{figure}[H]
  \centering
  \includegraphics[width=0.7\textwidth]{out/diagrams/documents/scan_qr/scan_qr}
  \caption{Diagramme de séquence de cas d'utilisation « Recherche a l'aide de code QR »}
  \label{fig:sequence_scan_qr}
\end{figure}

\subsubsection{Analyse détaillée}
\textbf{•	Diagramme de classe d'analyse de sprint "3" }

\begin{figure}[H]
  \centering
  \includegraphics[width=1\textwidth]{dca_sprint3}
  \caption{Diagramme de classe d'analyse de sprint 3}
  \label{fig:class_analyse_signatures3}
\end{figure}


\subsubsection{Conception}

Après la présentation des diagrammes d'analyse, nous avons présenté dans cette partie les diagrammes de conception.
Nous allons présenter dans cette partie les diagrammes de conception de sprint 3.
\newpage 
\begin{landscape}

\textbf{•	Diagramme de classe de conception de sprint 3 : « Gestion des documents »}

\begin{figure}[H]
  \centering
  \includegraphics[height=0.8\textheight]{dcc_spint3}
  \caption{Diagramme de classe de conception de sprint 3 : « Gestion des signatures »}
  \label{fig:class_diagram_signatures3}
\end{figure}
\end{landscape}
\newpage
\begin{figure}[H]
  \centering
  \includegraphics[width=1\textwidth]{out/diagrams/documents/sequence_preview/sequence_preview}
  \caption{Diagramme de séquence de conception de cas d'utilisation « Accéder à un document »}
  \label{fig:sequence_conception_preview_document}
\end{figure}
\begin{figure}[H]
  \centering
  \includegraphics[width=1\textwidth]{out/diagrams/documents/sequence_preview_files/sequence_preview_files}
  \caption{Diagramme de séquence de conception de cas d'utilisation « Consulter la liste des fichiers d'un document »}
  \label{fig:sequence_conception_previewFiles}
\end{figure}
\begin{figure}[H]
  \centering
  \includegraphics[width=1\textwidth]{out/diagrams/documents/sequence_preview_tasks/sequence_preview_tasks}
  \caption{Diagramme de séquence de conception de cas d'utilisation « Consulter la liste des tâches d'un document »}
  \label{fig:sequence_conception_previewTasks}
\end{figure}
\begin{figure}[H]
  \centering
  \includegraphics[width=1\textwidth]{out/diagrams/documents/sequence_favoris/sequence_favoris}
  \caption{Diagramme de séquence de conception de cas d'utilisation « Consulter la liste des documents en favoris »}
  \label{fig:sequence_conception_favoritDocument}
\end{figure}
\begin{figure}[H]
  \centering
  \includegraphics[width=1\textwidth]{out/diagrams/documents/sequence_transfer_task/sequence_transfer_task}
  \caption{Diagramme de séquence de conception de cas d'utilisation « Transferer une tâche »}
  \label{fig:sequence_conception_transferTask}
\end{figure}
\begin{figure}[H]
  \centering
  \includegraphics[width=1\textwidth]{out/diagrams/documents/sequence_traiter_task/sequence_traiter_task}
  \caption{Diagramme de séquence de conception de cas d'utilisation « Traiter une tâche »}
  \label{fig:sequence_conception_traiterTask}
\end{figure}
\begin{figure}[H]
  \centering
  \includegraphics[width=1\textwidth]{out/diagrams/documents/sequence_add_file/sequence_add_file}
  \caption{Diagramme de séquence de conception de cas d'utilisation « Ajoutez un fichier à un document »}
  \label{fig:sequence_conception_addFile}
\end{figure}
\begin{figure}[H]
  \centering
  \includegraphics[width=1\textwidth]{out/diagrams/documents/sequence_chercher/sequence_chercher}
  \caption{Diagramme de séquence de conception de cas d'utilisation « Chercher des documents »}
  \label{fig:sequence_conception_charcherDocument}
\end{figure}

\subsubsection{Réalisation et Tests}
\textbf{•	Réalisation }
Après la présentation des diagrammes d'analyse, nous avons présenté dans cette partie des captures d'écran de l'application.

\begin{figure}[H]
  \centering
  \includegraphics[width=0.6\textwidth]{realisation_documents1}
  \caption{Réalisation de la page des documents (Favoris et recherche(Simple et code QR))}
  \label{fig:realisation_documents1}
\end{figure}

\begin{figure}[H]
  \centering
  \includegraphics[width=0.6\textwidth]{realisation_documents2}
  \caption{Réalisation de la page de document (Fichier, Info et Tâches)}
  \label{fig:realisation_documents2}
\end{figure}

\begin{figure}[H]
  \centering
  \includegraphics[width=0.6\textwidth]{realisation_documents3}
  \caption{Réalisation de la page de document (Ajouter un fichier (Importer et Scanner))}
  \label{fig:realisation_documents3}
\end{figure}

\begin{figure}[H]
  \centering
  \includegraphics[width=0.6\textwidth]{realisation_documents4}
  \caption{Réalisation de la page de document (ajouter, transférer et traiter une tâche)}
  \label{fig:realisation_documents4}
\end{figure}

\textbf{•	Test unitaire} \\
Cette section du rapport se concentre sur les outils et les méthodologies utilisés pour l'exécution des tests unitaires, tels que les frameworks de test, les bibliothèques et les simulateurs.

Nous présenterons également la planification, l'exécution et l'évaluation des tests unitaires réalisés dans le cadre du troisième sprint de développement.

\textbf{•	Environnement de test unitaire}

La figure suivante présente l'environnement de test unitaire utilisé pour tester les composants de l'application.

\begin{figure}[H]
  \centering
  \includegraphics[width=0.5\textwidth]{testingtech}
  \caption{Environnement de test unitaire}
  \label{fig:environnementdetestunitaire}
\end{figure}
\begin{itemize}
  \item \textbf{JEST} : Jest est un framework de test JavaScript créé par Meta. Il est utilisé pour tester des applications JavaScript, notamment React, Vue.js, Angular et NodeJS.
  \item \textbf{PiniaTesting}: Pinia Testing est une bibliothèque de test pour Pinia, qui fournit des utilitaires pour tester les actions, les getters et les mutations.
  \item \textbf{Vue Test Utils}: Vue Test Utils est la bibliothèque de test officielle pour Vue.js. Il fournit des utilitaires qui facilitent l'écriture de tests unitaires pour les composants Vue.
\end{itemize}


\textbf{•	Cycle de développement du test unitaire}

La figure suivante présente le cycle de développement du test unitaire utilisé pour tester les composants de l'application.

\begin{figure}[H]
  \centering
  \includegraphics[width=0.3\textwidth]{diagramme_test}
  \caption{Cycle de développement du test unitaire}
  \label{fig:cyclededveloppementdutestunitaire}
\end{figure}

\textbf{•	Planification des tests unitaires}

\textbf{Note : } Jusqu'à présent, nous avons couvrés:
\begin{itemize}
  \item Le composant de tâche.
  \item Le panneau de confirmation de la tâche.
  \item Le composant des actions du document.
\end{itemize}

\textbf{•	Exécution des tests unitaires}

La figure suivante présente les résultats des tests unitaires réalisés dans le cadre du troisième sprint de développement.

\begin{figure}[H]
  \centering
  \includegraphics[width=0.8\textwidth]{capture_test_success_sprint_3}
  \caption{Résultats des tests unitaires}
  \label{fig:resultatstestunitaire}
\end{figure}

Comme le montre la figure ci-dessus, tous les tests unitaires ont réussi.


\subsection{Sprint Review:}
A la fin de ce sprint, nous avons planifié une réunion dans la société Neoledge avec le vise avis à Lille, France afin de vérifier notre démarche de travail par rapport au besoin de client tout en respectant le délai que nous avons prévu.
Nous avons fait une démonstration durant laquelle nous allons présenter notre incrément :
\begin{itemize}
  \item La consultation d'un document.
  \item La gestion des fichiers d'un document.
  \item La gestion des taches d'un document.
  \item La recherche d'un document.
  \item La consultation de la liste des documents en favoris.
  \item L'ajout des tests unitaires.
\end{itemize}

\subsection{Sprint Retrospective:}

Après la Sprint Review, nous avons réfléchi à des pistes pour améliorer la qualité et l'efficacité de notre application.\\
\noindent\textbf{•	Ce qui a bien passé :}
\begin{itemize}
  \item Nous avons terminé le sprint dans le délai.
  \item Nous avons bien appris le test unitaire dans VueJS.
\end{itemize}
\noindent\textbf{•	Ce qui s'est mal passé :}
Manque de documentation de l'ionic 


% SPrint 4 visualisation et signature d'un fichier
\section{Sprint 4 (Visualisation et signature d'un fichier)}
\subsection{Sprint Goal:}
L'objectif de ce sprint est de développer et mettre en place un système de visualisation et de signature d'un fichier permettant aux utilisateurs de consulter le fichier et le signer par plusieurs méthodes.


\subsection{Sprint Backlog « Visualisation et signature d'un fichier »:}

\begin{adjustwidth}{-1cm}{}
  % \usepackage{longtable}
    
    \begin{longtable}{|c|p{6cm}|c|p{6cm}|c|}
      \hline
      \textbf{ID} & \textbf{User story} & \textbf{ID}  & \textbf{Tâche} & \textbf{Durée} \\
      \hline
      \multirow{5}{*}{1} & \multirow{5}{6cm}{En tant qu'utilisateur, je veux visualiser des fichiers stockés dans le document, afin de consulter leur contenu sans avoir besoin d'une autre application ou d'un autre dispositif.} & 1.1 & Préparer les interfaces sur Figma. & \multirow{5}{*}{2 Jour} \\
      \cline{3-4}
      & & 1.2 & Développer l'interface de visualisation d'un fichier. & \\
      \cline{3-4}
      & & 1.3 & Intégrer la bibliothèque neo-pdf-viewer dans l'application. & \\
      \cline{3-4}
      & & 1.4 & Ajouter la fonctionnalité de zoom dans neo-pdf-viewer. & \\
      \cline{3-4}
      & & 1.5 & Prendre en charge les fichiers de type image. & \\
      \cline{1-5}

      \multirow{1}{*}{2} & En tant qu'utilisateur, je veux télécharger un fichier, afin d'avoir une copie locale du fichier sur mon appareil. & 2.1 & Développer une fonction qui permet de télécharger un fichier. & \multirow{1}{*}{2 Jour} \\
      \cline{1-5}

      \multirow{4}{*}{3} & \multirow{4}{6cm}{En tant qu'utilisateur, je souhaite signer des fichiers avec image, Je veux être en mesure de sélectionner une signature pré-enregistrée parmi celles que j'ai créées précédemment, en utilisant simplement un système de glisser déposer pour l'appliquer au document à signer.}& 3.1 & Préparer les interfaces sur Figma. & \multirow{4}{*}{2 Jour} \\
      \cline{3-4}
      & & 3.2 & Développer l'interface pour afficher la liste des signatures. & \\
      \cline{3-4}
      & & 3.3 & Développer une fonction qui permet d'afficher la liste des signatures. & \\
      \cline{3-4}
      & & 3.4 & Développer une fonction qui permet de signer par glisser et déposer une image dans un fichier. & \\
      \cline{1-5}

      \multirow{3}{*}{4} & En tant qu'utilisateur, je souhaite signer des fichiers a main. Je veux signer le document directement en utilisant une méthode de signature manuscrite, en dessinant ma signature. & 4.1 & Préparer les interfaces sur Figma. & \multirow{3}{*}{2 Jour} \\
      \cline{3-4}
      & & 4.2 & Développer une fonction qui permet de signer à main un fichier. & \\
      \cline{1-5}

      \multirow{1}{*}{5} & En tant qu'utilisateur, je souhaite déplacer les signatures non confirmées, afin de les positionner correctement.& 5.1 & Développer une fonction qui permet de modifier la position d'une signature d'un fichier. & \multirow{1}{*}{2 Jour} \\
      \cline{1-5}

      \multirow{2}{*}{6} & \multirow{2}{6cm}{En tant qu'utilisateur, je souhaite supprimer les signatures non confirmées, afin d'annuler la signature.}& 6.1 & Développer l'interface de suppression d'une signature. & \multirow{2}{*}{2 Jour} \\
      \cline{3-4}
      & & 6.2 & Développer une fonction qui permet de supprimer une signature d'un fichier. & \\
      \cline{1-5}

      \multirow{2}{*}{7} &\multirow{2}{6cm}{ En tant qu'utilisateur, je souhaite confirmer ou annuler la signature, afin de la valider ou de la supprimer.}& 7.1 & Développer l'interface de confirmation ou annulation de la signature. & \multirow{2}{*}{2 Jour} \\
      \cline{3-4}
      & & 7.2 & Développer une fonction qui permet de confirmer ou annuler la signature d'un fichier. & \\
      \cline{1-5}
      
  \hline
  \caption{Sprint backlog du Sprint 4}
  \label{tab:sprint-backlog-4}
\end{longtable}
\end{adjustwidth}

\subsection{Implémentation du Sprint 4}
\textbf{•	Diagramme de cas d'utilisation du sprint 4 : « Visualisation et signature d'un fichier »}

% add image
\begin{figure}[H]
  \centering
  \includegraphics[width=0.8\textwidth]{use_case_documents_sprint_4}
  \caption{Diagramme de cas d'utilisation du sprint 4 : « Visualisation et signature d'un fichier »}
  \label{fig:UseCaseDiagram4}
\end{figure}
\subsubsection{Analyse des besoins:}
\textbf{•	Description textuelle de cas d'utilisation « Visualiser un fichier  »}
\begin{longtable}{|p{5cm}|p{10cm}|}
\hline
\textbf{Cas d'utilisation}&Visualiser un fichier\\
\hline
\textbf{Acteurs}&Utilisateur\\
\hline
\textbf{Pré Condition}&Le fichier existe\\
\hline
\textbf{Post Condition}&Le fichier est visualisé\\
\hline
\textbf{Scénario Nominal}&
\vspace{-\baselineskip}
\begin{enumerate}
    \setcounter{enumi}{1}
  \item L'utilisateur clique sur le fichier.
  \item le système affiche le fichier.
\end{enumerate}\\
\hline
\textbf{Scénario Alternatif}&
\vspace{-\baselineskip}
\begin{enumerate}
    \setcounter{enumi}{2}
    \item Aucun résultat.
\end{enumerate}\\
\hline
\textbf{Scénario d'exception}&Erreur de connexion\\
\hline
\caption{Description textuelle de cas d'utilisation « Visualiser un fichier  »}
\label{tab:description-textuelle-de-cas-d-utilisation-visualiser-un-fichier}

\end{longtable}

\textbf{•	Description textuelle de cas d'utilisation « Télécharger un fichier »}
\begin{longtable}{|p{5cm}|p{10cm}|}
\hline
\textbf{Cas d'utilisation}&Télécharger un fichier\\
\hline
\textbf{Acteurs}&Utilisateur\\
\hline
\textbf{Pré Condition}&L'utilisateur est en cours de visualisation d'un fichier\\
\hline
\textbf{Post Condition}&Le fichier est téléchargé\\
\hline
\textbf{Scénario Nominal}&
\vspace{-\baselineskip}
\begin{enumerate}
    \setcounter{enumi}{1}
  \item L'utilisateur clique sur afficher les actions.
  \item Le système affiche les actions.
  \item L'utilisateur clique sur télécharger.
  \item Le système télécharge le fichier.
  \item Le système affiche un message de succès.
\end{enumerate}\\
\hline
\textbf{Scénario Alternatif}&
\vspace{-\baselineskip}
\begin{enumerate}
  \item [3.1] La fichier a des signatures non confirmées.
  \item [3.2] Le système demande si l'utilisateur veut télécharger le fichier sans les signatures.
  % second alternative
  \item [3.3.1] L'utilisateur clique sur oui.
  \item [3.3.2] Le système continue de la point 4 du scénario nominal.
  \item [3.3.1] L'utilisateur clique sur non.
  \item [3.3.2] Le système annule le téléchargement.
  \item [4.1] Le système n'arrive pas à télécharger le fichier.
  \item [4.2] Le système affiche un message d'erreur.

\end{enumerate}\\
\hline
\textbf{Scénario d'exception}&
\vspace{-\baselineskip}
\begin{enumerate}
  \item [3.1] le système n'a pas le droit de télécharger le fichier.
  \item [3.2] Le système affiche un message d'erreur.
\end{enumerate}\\
\hline
\caption{Description textuelle de cas d'utilisation « Télécharger un fichier »}
\label{tab:description-textuelle-de-cas-d-utilisation-telecharger-un-fichier}
\end{longtable}


\textbf{•	Description textuelle de cas d'utilisation « Signer par glisser et déposer une image dans un fichier »}

\begin{longtable}{|p{5cm}|p{10cm}|}
\hline
\textbf{Cas d'utilisation}&Signer par glisser et déposer une image dans un fichier\\
\hline
\textbf{Acteurs}&Utilisateur\\
\hline
\textbf{Pré Condition}&Le fichier existe\\
\hline
\textbf{Post Condition}&La signature est placée\\
\hline
\textbf{Scénario Nominal}&
\vspace{-\baselineskip}
\begin{enumerate}
    \setcounter{enumi}{1}
    \item L'utilisateur glisse ou clique sur le bouton du panel.
    \item Le système affiche le panel qui contient la liste des signatures.
    \item L'utilisateur glisse une signature.
    \item L'utilisateur dépose la signature dans le fichier.
    \item Le système place la signature dans le fichier.
\end{enumerate}\\
\hline
\textbf{Scénario Alternatif}&
\vspace{-\baselineskip}
\begin{enumerate}
    \item [4.1] L'utilisateur dépose la signature hors du fichier.
    \item [4.2] Le système ne place pas la signature dans le fichier.
\end{enumerate}\\
\hline
\caption{Description textuelle de cas d'utilisation « Signer par glisser et déposer une image dans un fichier »}
\label{tab:description-textuelle-de-cas-d-utilisation-signer-par-glisser-et-deposer-une-image-dans-un-fichier}
\end{longtable}

\textbf{•	Description textuelle de cas d'utilisation « Signer à main un fichier »}

\begin{longtable}{|p{5cm}|p{10cm}|}
\hline
\textbf{Cas d'utilisation}&Signer à main un fichier\\
\hline
\textbf{Acteurs}&Utilisateur\\
\hline
\textbf{Pré Condition}&Le fichier existe\\
\hline
\textbf{Post Condition}&La signature est tracée\\
\hline
\textbf{Scénario Nominal}&
\vspace{-\baselineskip}
\begin{enumerate}
    \setcounter{enumi}{1}
  \item L'utilisateur fait une longue clique sur le fichier
  \item L'utilisateur clique sur afficher les actions.
  \item Le système affiche les actions.
  \item L'utilisateur clique sur le bouton signature à main.
  \item L'utilisateur trace sa signature.
  \item Le système affiche les 2 boutons confirmer et annuler
  \item L'utilisateur clique sur le bouton confirmer
  \item Le système place la signature dans le fichier.
\end{enumerate}\\
\hline
\textbf{Scénario Alternatif}&
\vspace{-\baselineskip}
\begin{enumerate}
    \item [6.1]L'utilisateur clique sur le bouton annuler
    \item [6.2]Le tracé de la signature est effacé.
\end{enumerate}\\
\hline
\caption{Description textuelle de cas d'utilisation « Signer à main un fichier »}
\label{tab:description-textuelle-de-cas-d-utilisation-signer-a-main-un-fichier}
\end{longtable}


\textbf{•	Description textuelle de cas d'utilisation « Modifier la position d'une signature d'un fichier »}

\begin{longtable}{|p{5cm}|p{10cm}|}
\hline
\textbf{Cas d'utilisation}&Modifier la position d'une signature d'un fichier\\
\hline
\textbf{Acteurs}&Utilisateur\\
\hline
\textbf{Pré Condition}&La signature est placée et l'utilisateur n'a pas encore confirmer les changements\\
\hline
\textbf{Post Condition}&La position de la signature est modifiée\\
\hline
\textbf{Scénario Nominal}&
\vspace{-\baselineskip}
\begin{enumerate}
    \setcounter{enumi}{1}
    \item L'utilisateur clique sur la signature.
    \item L'utilisateur deplace et dépose la signature.
    \item Le système modifie la position de la signature.
\end{enumerate}\\
\hline
\textbf{Scénario Alternatif}&
\vspace{-\baselineskip}
\begin{enumerate}
    \item [2.1]L'utilisateur dépose la signature hors du fichier.
    \item [2.2]Le système annule les changements.
\end{enumerate}\\
\hline
\caption{Description textuelle de cas d'utilisation « Modifier la position d'une signature d'un fichier »}
\label{tab:description-textuelle-de-cas-d-utilisation-modifier-la-position-d-une-signature-d-un-fichier}
\end{longtable}

\textbf{•	Description textuelle de cas d'utilisation « Supprimer une signature d'un fichier  »}

\begin{longtable}{|p{5cm}|p{10cm}|}
\hline
\textbf{Cas d'utilisation}&Supprimer une signature d'un fichier \\
\hline
\textbf{Acteurs}&Utilisateur\\
\hline
\textbf{Pré Condition}&La signature est placée et l'utilisateur n'a pas encore confirmer les changements\\
\hline
\textbf{Post Condition}&La signature est supprimée\\
\hline
\textbf{Scénario Nominal}&
\vspace{-\baselineskip}
\begin{enumerate}
    \setcounter{enumi}{1}
    \item L'utilisateur déplace la signature dans la zone de suppression.
    \item Le système supprime la signature d'un fichier
\end{enumerate}\\
\hline
\textbf{Scénario Alternatif}&
\vspace{-\baselineskip}
\begin{enumerate}
    \item [1.1]L'utilisateur dépose la signature hors du fichier.
    \item [1.2]Le système annule les changements.
\end{enumerate}\\
\hline
\caption{Description textuelle de cas d'utilisation « Supprimer une signature d'un fichier  »}
\label{tab:description-textuelle-de-cas-d-utilisation-supprimer-une-signature-d-un-fichier}
\end{longtable}


\textbf{•	Description textuelle de cas d'utilisation « Confirmer ou annuler les changements d'un fichier »}

\begin{longtable}{|p{5cm}|p{10cm}|}
\hline
\textbf{Cas d'utilisation}&Confirmer ou annuler les changements d'un fichier\\
\hline
\textbf{Acteurs}&Utilisateur\\
\hline
\textbf{Pré Condition}&Le fichier est modifié\\
\hline
\textbf{Post Condition}&Les changements sont enregistrés ou annulés\\
\hline
\textbf{Scénario Nominal}&
\vspace{-\baselineskip}
\begin{enumerate}
    \setcounter{enumi}{1}
    \item L'utilisateur clique sur afficher les actions.
    \item Le système affiche les actions.
    \item L'utilisateur clique sur le bouton confirmer.
    \item Le système enregistre les changements et affiche un message des succès.
\end{enumerate}\\
\hline
\textbf{Scénario Alternatif}&
\vspace{-\baselineskip}
\begin{enumerate}
    \item [3.1]L'utilisateur clique sur le bouton annuler.
    \item [3.2]Le système annule les changements.
\end{enumerate}\\
\hline
\textbf{Scénario d'exception}&Erreur de connexion\\
\hline
\caption{Description textuelle de cas d'utilisation « Confirmer ou annuler les changements d'un fichier »}
\label{tab:description-textuelle-de-cas-d-utilisation-confirmer-ou-annuler-les-changements-d-un-fichier}
\end{longtable}


Les maquettes suivantes représentent les interfaces de chaque cas d'utilisation de cette sprint. \\

\begin{figure}[H]
  \centering
  \includegraphics[width=1\textwidth]{design_file}
  \caption{Maquette de l'affichage d'un fichier}
  \label{fig:design_file_preview}
\end{figure}

Dans les diagrammes de séquence suivants, nous allons détailler les cas d'utilisation de cette sprint.\\
\begin{figure}[H]
  \centering
  \includegraphics[width=0.7\textwidth]{out/diagrams/sprint4/view_file/view_file}
  \caption{Diagramme de séquence de cas d'utilisation « Visualiser un fichier   »}
  \label{fig:sequence_view_file}
\end{figure}

% Télécharger un fichier 
\begin{figure}[H]
  \centering
  \includegraphics[width=0.7\textwidth]{out/diagrams/sprint4/download_file/download_file}
  \caption{Diagramme de séquence de cas d'utilisation « Télécharger un fichier   »}
  \label{fig:sequence_download_file}
\end{figure}
% Signer par glisser et déposer une image dans un fichier 
\begin{figure}[H]
  \centering
  \includegraphics[width=0.8\textwidth]{out/diagrams/sprint4/sign_by_image/sign_by_image}
  \caption{Diagramme de séquence de cas d'utilisation « Signer par glisser et déposer une image dans un fichier   »}
  \label{fig:sequence_sign_by_image}
\end{figure}

% Signer à main un fichier Modifier la position d'une signature d'un fichier 
\begin{figure}[H]
  \centering
  \includegraphics[width=0.6\textwidth]{out/diagrams/sprint4/sign_by_hand/sign_by_hand}
  \caption{Diagramme de séquence de cas d'utilisation « Signer à main un fichier   »}
  \label{fig:sequence_sign_by_hand}
\end{figure}
% Modifier la position d'une signature d'un fichier 
\begin{figure}[H]
  \centering
  \includegraphics[width=0.6\textwidth]{out/diagrams/sprint4/move_signature/move_signature}
  \caption{Diagramme de séquence de cas d'utilisation « Modifier la position d'une signature d'un fichier »}
  \label{fig:sequence_move_signature}
\end{figure}

% Supprimer une signature d'un fichier 
\begin{figure}[H]
  \centering
  \includegraphics[width=0.7\textwidth]{out/diagrams/sprint4/delete_signature/delete_signature}
  \caption{Diagramme de séquence de cas d'utilisation « Supprimer une signature d'un fichier »}
  \label{fig:sequence_delete_signature_sp4}
\end{figure}
% Confirmer ou annuler les changements d'un fichier 
\begin{figure}[H]
  \centering
  \includegraphics[width=0.7\textwidth]{out/diagrams/sprint4/save_cancel_siganture/save_cancel_siganture}
  \caption{Diagramme de séquence de cas d'utilisation « Confirmer ou annuler les changements d'un fichier »}
  \label{fig:sequence_save_cancel_siganture}
\end{figure}



\subsubsection{Analyse détaillée}

\textbf{•	Diagramme de classe d'analyse de sprint "4" }

\begin{figure}[H]
  \centering
  \includegraphics[width=1\textwidth]{dca_sprint4}
  \caption{Diagramme de classe d'analyse de sprint 4}
  \label{fig:class_analyse_4}
\end{figure}



\subsubsection{Conception}

Après la présentation des diagrammes d'analyse, nous avons présenté dans cette partie les diagrammes de conception.
Nous allons présenter dans cette partie les diagrammes de conception de sprint 4.
\newpage 
\begin{landscape}

\textbf{•	Diagramme de classe de conception de sprint 4 : « Visualisation et signature d'un fichier »}

\begin{figure}[H]
  \centering
  \includegraphics[height=0.8\textheight]{dcc_spint4}
  \caption{Diagramme de classe de conception de sprint 4 : « Visualisation et signature d'un fichier »}
  \label{fig:class_diagram_4}
\end{figure}
\end{landscape}
\newpage


\begin{figure}[H]
  \centering
  \includegraphics[width=0.7\textwidth]{out/diagrams/sprint4/sequence_view_file/sequence_view_file}
  \caption{Diagramme de séquence de conception de cas d'utilisation « Visualiser un fichier »}
  \label{fig:sequence_conception_view_file}
\end{figure}

\begin{figure}[H]
  \centering
  \includegraphics[width=0.7\textwidth]{out/diagrams/sprint4/sequence_download_file/sequence_download_file}
  \caption{Diagramme de séquence de conception de cas d'utilisation « Télécharger un fichier »}
  \label{fig:sequence_conception_download_file}
\end{figure}

\begin{figure}[H]
  \centering
  \includegraphics[width=0.7\textwidth]{out/diagrams/sprint4/sequence_sign_by_image/sequence_sign_by_image}
  \caption{Diagramme de séquence de conception de cas d'utilisation « Signer par glisser et déposer une image dans un fichier »}
  \label{fig:sequence_conception_sign_by_image}
\end{figure}

\begin{figure}[H]
  \centering
  \includegraphics[width=0.7\textwidth]{out/diagrams/sprint4/sequence_sign_by_hand/sequence_sign_by_hand}
  \caption{Diagramme de séquence de conception de cas d'utilisation « Signer à main un fichier »}
  \label{fig:sequence_conception_sign_by_hand}
\end{figure}

\begin{figure}[H]
  \centering
  \includegraphics[width=1\textwidth]{out/diagrams/sprint4/sequence_move_signature/sequence_move_signature}
  \caption{Diagramme de séquence de conception de cas d'utilisation « Modifier la position d'une signature d'un fichier »}
  \label{fig:sequence_conception_move_signature}
\end{figure}

\begin{figure}[H]
  \centering
  \includegraphics[width=1\textwidth]{out/diagrams/sprint4/sequence_delete_signature/sequence_delete_signature}
  \caption{Diagramme de séquence de conception de cas d'utilisation « Supprimer une signature d'un fichier »}
  \label{fig:sequence_conception_delete_signature_from_foile}
\end{figure}

\begin{figure}[H]
  \centering
  \includegraphics[width=1\textwidth]{out/diagrams/sprint4/sequence_save_cancel_siganture/sequence_save_cancel_siganture}
  \caption{Diagramme de séquence de conception de cas d'utilisation « Confirmer ou annuler les changements d'un fichier »}
  \label{fig:sequence_conception_save_cancel_siganture}
\end{figure}



\subsubsection{Réalisation}

\begin{figure}[H]
  \centering
  \includegraphics[width=0.45\textwidth]{realisation_file1}
  \caption{Captures de l'affichage d'un fichier (par page ou par liste)}
  \label{fig:realisation_file_1}
\end{figure}
\begin{figure}[H]
  \centering
  \includegraphics[width=0.5\textwidth]{realisation_file2}
  \caption{Captures de l'affichage d'un fichier (partie signature)}
  \label{fig:realisation_file_2}
\end{figure}
\begin{figure}[H]
  \centering
  \includegraphics[width=0.5\textwidth]{realisation_file3}
  \caption{Captures de l'affichage d'un fichier (validation/annulation des signatures et téléchargement)}
  \label{fig:realisation_file_3}
\end{figure}

\begin{figure}[H]
  \begin{subfigure}[t]{0.4\textwidth}
  \centering
  \includegraphics[width=\linewidth]{prendre_en_charge_image}
  \caption{Capture de la prise en charge de l'affichage d'une image.}
  \label{fig:prendre_en_charge_image}
  \end{subfigure}\hfill
  \begin{subfigure}[t]{0.4\textwidth}
  \centering
  \includegraphics[width=\linewidth]{zoom}
  \caption{Capture de l'affichage d'un fichier (zoom).}
  \label{fig:zoom}
  \end{subfigure}
  \caption{Captures de l'amélioration de neo-pdf-viewer.}
\label{fig:zoom_and_image}
\end{figure}

Voir annexe pour plus de détails sur l'amélioration de neo-pdf-viewer.


\subsection{Sprint review:}


A la fin de ce sprint, nous avons planifié une autre réunion dans la société Neoledge  afin de vérifier notre démarche de travail par rapport au besoin de client tout en respectant le délai que nous avons prévu.

Nous avons fait une démonstration durant laquelle nous allons présenter notre incrément :
\begin{itemize}
  \item La visualisation d'un fichier.
  \item La signature à main d'un fichier.
  \item La signature par glisser et déposer une image dans un fichier.
  \item La modification de la position d'une signature d'un fichier.
  \item La suppression d'une signature d'un fichier.
  \item La confirmation ou l'annulation de la signature d'un fichier
\end{itemize}

\subsection{Sprint retrospective:}

Après la Sprint Review, nous avons réfléchi à des pistes pour améliorer la qualité et l'efficacité de notre application.


\noindent\textbf{•	Ce qui s'est bien passé :}
Nous avons terminé le sprint dans le délai.
\noindent\textbf{•	Ce qui s'est mal passé :}
\begin{itemize}
  \item Manque de documentation de l'ionic
  \item Difficultés de déposer la signature à la position exacte.

\end{itemize}

