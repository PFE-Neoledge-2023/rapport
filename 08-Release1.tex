\chapter{Release 1}
\addcontentsline{toc}{chapter}{Release 1}
\markboth{Release 1}{Release 1}
\renewcommand\fbox{\fcolorbox{blue}{white}}
\label{chap:release1}
% section starts from 1 
%\minitoc

\section*{Introduction}

Ce chapitre présente la première version de notre application. Il est composé de cinq sprints qui ont été réalisés en 10 semaines. Nous avons commencé par la conception de l'application, puis nous avons implémenté les fonctionnalités de base de l'application tels que la gestion des documents et des signatures. Dans ce chapitre, nous décrirons en détail les fonctionnalités de chaque sprint, les défis que nous avons rencontrés et les solutions que nous avons apportées pour les surmonter.

Release 1 : (Du 8 Février Au 19 Avril)

\fbox{\begin{minipage}{30em}
  \textbf{Organisation des sprints :} \\
  Cette release contient les quatre sprints:
  \begin{itemize}
    \item \textbf{Sprint 1:} Préparation de l'environnement du travail et étude de la solution.
    \item \textbf{Sprint 2:} Gestion des signatures.
    \item \textbf{Sprint 3:} Gestion des documents.
    \item \textbf{Sprint 4:} Visualisation et signature de fichiers.
  \end{itemize}
\end{minipage}}

\section{Sprint 1 (Préparation de l'environnement du travail et étude de la solution)}

\subsection{Sprint Goal}
L'objectif de ce sprint est de préparer l'environnement de travail et d'étudier la solution ainsi que les technologies à utiliser.

\subsection{Sprint Backlog}

\begin{adjustwidth}{-1cm}{}
  % \usepackage{longtable}
    
    \begin{longtable}{|c|p{6cm}|c|p{6cm}|c|}
      % \centering
      \hline
      \textbf{ID} & \textbf{User story} & \textbf{ID}  & \textbf{Tâche} & \textbf{Durée} \\
      \hline

    1 & En tant que membre de l'équipe scrum, je
    souhaite comprendre pleinement la fonctionnalité du système Elise afin de reproduire sa logique dans une application mobile. &  1.1 &suivre une formation préparée par la société sur Elise.&3 Jours\\
    \cline{1-5}
    2 & \multirow{5}{6cm}{En tant que scrum team, je souhaite me former au développement mobile en utilisant les technologies Ionic Vue et Capacitor. Cette formation doit me permettre de maîtriser les compétences essentielles pour créer des applications mobiles multiplateformes de qualité.}  &  2.1 &Suivre une formation sur Youtube qui explique les notions de base d'Ionic.&\multirow{4}{2cm}{3.5 Jours}\\
    \cline{3-4}
    &  &  2.2 &Suivre une formation sur Youtube qui explique les notions de base du Capacitor.&\\
    \cline{3-4}
    &  &  2.3 &Suivre une formation sur Youtube qui explique les notions de base du Soap.&\\
    \cline{3-4}
    &  &  2.4 &Suivre une formation sur Youtube qui explique les notions de base du .NET core 6.&\\
    \cline{1-5}
    3 & En tant que membre de l'équipe scrum, je souhaite installer et configurer l'environnement de développement. &  3.1 &Installer VS code.&\multirow{7}{2cm}{0.5 Jours}\\
    \cline{3-4}
    &  &  3.2 &Installer Android Studio.&\\
    \cline{3-4}
    &  &  3.3 &Installer .NET Core 6 .&\\
    \cline{3-4}
    &  &  3.4 &Installer Ionic Version6.&\\
    \cline{3-4}
    &  &  3.5 &Installer Vue Js Version 3.&\\
    \cline{3-4}
    &  &  3.6 &Installer Capacitor Version 4.&\\
    \cline{3-4}
    &  &  3.7 &Installer SoapUi.&\\
    \cline{1-5}
    4 & En tant que développeur, je veux développer des prototypes de l'application mobile afin de tester les fonctionnalités et de valider les choix techniques &  4.1 &\textbf{Application 1} Développer une application mobile qui consomme le webservice soap pour afficher les données de la météo.&\multirow{4}{2cm}{4 Jours}\\
    \cline{3-4}
    &  &  4.2 &\textbf{Application 2} Développer une application mobile qui permet la création d'une signature.&\\
    \cline{3-4}
    &  &  4.3 &\textbf{Application 3} Développer une application mobile qui permet de visualiser un fichier PDF.&\\
    \cline{3-4}
    &  &  4.4 &\textbf{Application 4} Développer une application mobile qui permet de signer un fichier PDF en utilisant les deux solutions 2 et 3.&\\
    \cline{1-5}
    % S'authentifier
    5 & En tant que développeur, je veux développer une application mobile qui permet à l'utilisateur de s'authentifier afin d'accéder à l'application. & 
    
    5.1 &Initialiser un projet Ionic Vue.&\multirow{3}{2cm}{3 Jours}\\
    \cline{3-4}
    &  &  5.2 & Créer une page d'authentification.&\\
    \cline{3-4}
    &  &  5.3 & Développer la fonctionnalité d'authentification.&\\

  \hline

  \caption{Sprint Backlog du Sprint 1}
  \label{tab:sprint-backlog-1}
\end{longtable}
\end{adjustwidth}
\subsection{Sprint Review}
Suite à cette Technical Story, nous avons préparé notre environnement de travail où nous aurons les possibilités de traiter les prochains sprints.

\subsection{Sprint Retrospective}

\begin{itemize}
  \item \textbf{Ce qui a bien fonctionné :}
  \begin{itemize}
    \item Nous avons pu suivre les formations préparées par la société.
    \item Nous avons pu suivre les formations sur Youtube.
    \item Nous avons pu installer les outils nécessaires pour le développement.
    \item Nous avons bien mis nos connaissances en pratique dans certains projets préparatoires (Application 1, 2, 3 et 4).
  \end{itemize}
  \item \textbf{Ce qui n'a pas bien fonctionné :}
  
  Nous avons remarqué que le temps de formation est très réduit, nous avons donc décidé de suivre des formations sur Youtube pour nous former sur les technologies à utiliser en plus de la formation préparée par la société.
\end{itemize}

\section{Sprint 2 (Gestion des signatures)}

\subsection{Sprint Goal}

L'objectif de ce sprint est de développer et mettre en place un système de gestion des signatures permettant aux utilisateurs d'ajouter et de visualiser facilement les signatures crée, tout en garantissant la sécurité.

\subsection{Sprint Backlog}


\begin{adjustwidth}{-1cm}{}
  % \usepackage{longtable}
    
    \begin{longtable}{|c|p{6cm}|c|p{6cm}|c|}
      % \centering
      \hline
      \textbf{ID} & \textbf{User story} & \textbf{ID}  & \textbf{Tâche} & \textbf{Durée} \\
      \hline
      \multirow{2}{*}{1} & En tant qu'utilisateur, je veux créer une signature numérique en dessinant ma signature à l'aide de mon doigt ou de mon stylet sur l'écran tactile de mon appareil mobile, afin de la réutiliser facilement lors de la signature des documents. & 1.1 & Préparer les interfaces sur Figma. & \multirow{3}{*}{2.5 Jour} \\
      \cline{3-4}
      & & 1.2 & Développer l'interface de création de signature. & \\
      \cline{3-4}
      & & 1.3 & Développer la fonction qui permet de créer une signature. & \\
      \cline{1-5}
      \multirow{2}{*}{2} & En tant qu'utilisateur, je veux lister les signatures que j'ai créées, afin de les visualiser facilement. & 2.1 & Préparer les interfaces sur Figma. & \multirow{3}{*}{1 Jour} \\
      \cline{3-4}
      & & 2.2 & Développer l'interface de listage des signatures. & \\
      \cline{3-4}
      & & 2.3 & Développer la fonction qui permet de lister les signatures. & \\
      \cline{1-5}
      \multirow{2}{*}{3} & En tant qu'utilisateur, je veux modifier le nom, la description et si elle est par défaut d'une signature que j'ai créée auparavant, afin de la mettre à jour ou de la rendre plus facile à identifier. & 3.1 & Préparer les interfaces sur Figma. & \multirow{3}{*}{0.5 Jour} \\
      \cline{3-4}
      & & 3.2 & Développer l'interface de modification des signatures. & \\
      \cline{3-4}
      & & 3.3 & Développer la fonction qui permet de modifier une signature. & \\
      \cline{1-5}
      \multirow{2}{*}{3} & En tant qu'utilisateur, je veux supprimer une signature que j'ai créée, afin de ne pas utiliser une signature
      obsolète ou inexacte. & 3.1 & Préparer les interfaces sur Figma. & \multirow{3}{*}{0.5 Jour} \\
      \cline{3-4}
      & & 3.2 & Développer l'interface de suppression des signatures. & \\
      \cline{3-4}
      & & 3.3 & Développer la fonction qui permet de supprimer une signature. & \\
      \cline{1-5}
      \multirow{2}{*}{4} & En tant qu'utilisateur, je veux visualiser une signature que j'ai créée, pour m'assurer qu'elle est correcte. & 4.1 & Préparer les interfaces sur Figma. & \multirow{3}{*}{2 Jour} \\
      \cline{3-4}
      & & 4.2 & Développer l'interface de visualisation des signatures. & \\
      \cline{3-4}
      & & 4.3 & Développer la fonction qui permet de visualiser une signature. & \\
      \cline{1-5}
      \multirow{2}{*}{5} & En tant qu'utilisateur, je veux ajouter ma signature en utilisant la camera de mon appareil ou une image, afin de gagner du temps et de faciliter le processus de signature. & 5.1 & Préparer les interfaces sur Figma. & \multirow{3}{*}{5 Jour} \\
      \cline{3-4}
      & & 5.2 & Développer l'interface d'ajout de signature. & \\
      \cline{3-4}
      & & 5.3 & Développer la fonction qui permet d'ajouter une signature a l'aide de la camera. & \\
      \cline{3-4}
      & & 5.4 & Développer la fonction qui permet d'ajouter une signature a l'aide d'une image. & \\
      \cline{1-5}

  \hline
  \caption{Sprint backlog du Sprint 2}
  \label{tab:sprint-backlog-2}
\end{longtable}
\end{adjustwidth}

\subsection{Implémentation du Sprint 2}
\textbf{•	Diagramme de cas d'utilisation du sprint 2 : « Gestion des signatures »}

% add image
\begin{figure}[H]
  \centering
  \includegraphics[width=0.8\textwidth]{use_case_signature_sprint_2}
  \caption{Diagramme de cas d'utilisation du sprint 2 : « Gestion des signatures »}
  \label{fig:UseCaseDiagram}
\end{figure}

\subsubsection{Analyse des besoins:}
\textbf{•	Description textuelle de cas d'utilisation « Créer une signature »}

\begin{longtable}{|p{5cm}|p{10cm}|}
\hline
\textbf{Cas d'utilisation}&Créer une signature\\
\hline
\textbf{Acteurs}&Utilisateur\\
\hline
\textbf{Pré Condition}&L'utilisateur doit être authentifié\\
\hline
\textbf{Post Condition}&Création d'une signature\\
\hline
\textbf{Scénario Nominal}&
\vspace{-\baselineskip}
\begin{enumerate}
    \setcounter{enumi}{1}
  \item L'utilisateur dessine sa signature sur le pad.
  \item L'utilisateur clique sur le bouton enregistrer.
  \item Le système affiche un panneau d'ajout de nom.
  \item L'utilisateur entre le nom de la signature.
  \item L'utilisateur clique sur le bouton enregistrer.
  \item Le système enregistre la signature.
  \item Le système affiche un message de succès.
  \item Le system cache le panneau d'ajout de nom.
\end{enumerate}\\
\hline
\textbf{Scénario Alternatif}&
\vspace{-\baselineskip}
\begin{enumerate}
      \item [2.1] L'utilisateur ne dessine pas sa signature sur le pad.
      \item [2.2] Le système affiche un message d'erreur pour s'assurer de dessiner la signature.
      \item [5.1] L'utilisateur ne donne pas de nom à la signature.
      \item [5.2] Le système affiche un message d'erreur pour s'assurer de donner un nom à la signature.
\end{enumerate}\\
\hline
\textbf{Scénario d'exception}&Erreur de connexion\\
\hline
\caption{Description textuelle du diagramme de cas d'utilisation « Créer une signature »}
\label{tab:use_case_create_signature}
\end{longtable}

% \textbf{Terminologie paragraphe :} \\
% \textbf{OCR :} signifie Optical Character Recognition (reconnaissance optique de caractères en français). Il s'agit d'un processus de conversion d'images numérisées de textes en fichiers éditables et interprétables par des ordinateurs.
  

\textbf{•	Description textuelle de cas d'utilisation « Afficher les signatures »}

\begin{longtable}{|p{5cm}|p{10cm}|}
\hline
\textbf{Cas d'utilisation}&Afficher les signatures\\
\hline
\textbf{Acteurs}&Utilisateur \\
\hline
\textbf{Pré Condition}&L'utilisateur doit être authentifié\\
\hline
\textbf{Post Condition}&Affichage des signatures\\
\hline
\textbf{Scénario Nominal}&
\vspace{-\baselineskip}
\begin{enumerate}
    \setcounter{enumi}{1}
    \item L'utilisateur clique sur le bouton « Signatures ».
    \item Le système affiche la liste des signatures.
\end{enumerate}\\
\hline
\textbf{Scénario alternatif}&
\begin{enumerate}
  \item [2.1] L'utilisateur n'a pas de signature.
  \item [2.2] Le système affiche un message pour l'inviter à créer une signature.
\end{enumerate}\\
\hline
\textbf{Scénario d'exception}&Erreur de connexion\\
\hline
\caption{Description textuelle du diagramme de cas d'utilisation « Afficher les signatures »}
\label{tab:use_case_view_signature}
\end{longtable}

\textbf{•	Description textuelle de cas d'utilisation « Modifier une signature »}

\begin{longtable}{|p{5cm}|p{10cm}|}
\hline
\textbf{Cas d'utilisation}&Modifier une signature\\
\hline
\textbf{Acteurs}&Utilisateur\\
\hline
\textbf{Pré Condition}&L'utilisateur doit être authentifié et avoir des signatures\\
\hline
\textbf{Post Condition}&Modification d'une signature\\
\hline
\textbf{Scénario Nominal}&
\vspace{-\baselineskip}
\begin{enumerate}
    \setcounter{enumi}{1}
  \item L'utilisateur clique sur le bouton modifier.
  \item Le système affiche un panneau de modification.
  \item L'utilisateur modifie le nom, description ou si la signature est par défaut.
  \item L'utilisateur clique sur le bouton enregistrer.
  \item Le système enregistre la modification.
  \item Le système affiche un message de succès.
  \item Le system cache le panneau de modification.
\end{enumerate}\\
\hline
\textbf{Scénario Alternatif}&
\vspace{-\baselineskip}
\begin{enumerate}
      \item [4.1] L'utilisateur ne remplit pas le champ de nom.
      \item [4.2] Le système affiche un message d'erreur.
\end{enumerate}\\
\hline
\textbf{Scénario d'exception}&Erreur de connexion\\
\hline
\caption{Description textuelle du diagramme de cas d'utilisation « Créer une signature »}
\label{tab:use_case_update_signature}
\end{longtable}


\textbf{•	Description textuelle de cas d'utilisation « Supprimer une signature »}

\begin{longtable}{|p{5cm}|p{10cm}|}
\hline
\textbf{Cas d'utilisation}&Supprimer une signature\\
\hline
\textbf{Acteurs}&Utilisateur \\
\hline
\textbf{Pré Condition}&L'utilisateur doit être authentifié et avoir des signatures.\\
\hline
\textbf{Post Condition}&Suppression  d'une signature\\
\hline
\textbf{Scénario Nominal}&
\vspace{-\baselineskip}
\begin{enumerate}
    \setcounter{enumi}{1}
    \item L'utilisateur clique sur le bouton supprimer.
    \item Le système affiche une alerte de vérification.
    \item L'utilisateur clique sur le bouton confirme.
    \item La signature est supprimer de la base.
    \item Le système affiche un message de succès.
\end{enumerate}\\
\hline
\textbf{Scénario d'exception}&Erreur de connexion\\
\hline
\caption{Description textuelle du diagramme de cas d'utilisation « Supprimer une signature »}
\label{tab:use_case_delete_signature}
\end{longtable}


\textbf{•	Description textuelle de cas d'utilisation « Visualiser une signature »}

\begin{longtable}{|p{5cm}|p{10cm}|}
\hline
\textbf{Cas d'utilisation}&Visualiser une signature\\
\hline
\textbf{Acteurs}&Utilisateur \\
\hline
\textbf{Pré Condition}&L'utilisateur doit être authentifié et avoir des signatures.\\
\hline
\textbf{Post Condition}&Visualisation d'une signature\\
\hline
\textbf{Scénario Nominal}&
\vspace{-\baselineskip}
\begin{enumerate}
    \setcounter{enumi}{1}
    \item L'utilisateur clique sur le bouton visualiser.
    \item Le système affiche la signature.
\end{enumerate}\\
\hline
\textbf{Scénario d'exception}&Erreur de connexion\\
\hline
\caption{Description textuelle du diagramme de cas d'utilisation « Visualiser une signature »}
\label{tab:use_case_view_single_signature}
\end{longtable}


\textbf{•	Description textuelle de cas d'utilisation « Créer une signature a l'aide du caméra ou d'image »}
\begin{longtable}{|p{5cm}|p{10cm}|}
\hline
\textbf{Cas d'utilisation}&Créer une signature a l'aide du caméra ou d'image\\
\hline
\textbf{Acteurs}&Utilisateur \\
\hline
\textbf{Pré Condition}&L'utilisateur doit être authentifié.\\
\hline
\textbf{Post Condition}&Création d'une signature\\
\hline
\textbf{Scénario Nominal}&
\vspace{-\baselineskip}
\begin{enumerate}
    \setcounter{enumi}{1}
    \item L'utilisateur clique sur le bouton Importer.
    \item Le système affiche une fenêtre pour choisir ou prendre une image.
    \item L'utilisateur choisit l'image.
    \item Le système affiche l'image pour que l'utilisateur puisse la redimensionner.
    \item L'utilisateur redimensionne l'image.
    \item Le système récupère la signature de la partie de l'image choisie et l'affiche pour que l'utilisateur puisse la modifier.
    \item L'utilisateur modifie la signature.
    \item Le systeme procède à la création de la signature. // TODO !!
\end{enumerate}\\
\hline
\textbf{Scénario alternatif}&
\begin{enumerate}
  \item [2.1] L'utilisateur choisit une fichier qui n'est pas une image.
  \item [2.2] Le système affiche un message d'erreur.
  \item [4.1] L'utilisateur annule la redimension.
  \item [4.2] Le scenario reprend à l'étape 2.
\end{enumerate}\\
\hline
\textbf{Scénario d'exception}&Erreur de connexion\\
\hline
\caption{Description textuelle du diagramme de cas d'utilisation « Créer une signature a l'aide du caméra ou d'image »}
\label{tab:use_case_create_signature_from_image}
\end{longtable}

Pour créer notre plateforme, nous avons pensé qu'il était nécessaire de réaliser une maquette. Les maquettes nous aident à concevoir des interfaces qui répondent aux attentes et aux besoins du client. Elles permettent également de s'assurer que les besoins du client sont adaptés au projet. Nous avons réalisé un prototype de notre système, comme le montrent les figures ci-dessous. Tout au long de notre projet, nous présenterons des maquettes.

\begin{figure}[H]
  \centering
  \includegraphics[width=0.7\textwidth]{design_signatures}
  \caption{Maquette de la page des signatures}
  \label{fig:design_signatures}
\end{figure}

\begin{figure}[H]
  \centering
  \includegraphics[width=0.7\textwidth]{preview_delete_signature}
  \caption{Maquette de la page de visualisation et de suppression d'une signature}
  \label{fig:design_preview_delete_signature}
\end{figure}

Pour avoir une représentation temporelle des interactions entre les objets de notre système et de la chronologie des messages échangés entre eux et avec les acteurs nous avons réalisé les diagrammes de séquence représentés ci-dessous

\begin{figure}[H]
  \centering
  \includegraphics[width=0.7\textwidth]{out/diagrams/signatures/create/create_signature}
  \caption{Diagramme de séquence de cas d'utilisation « Créer une signature  »}
  \label{fig:sequence_create_signature}
\end{figure}

\begin{figure}[H]
  \centering
  \includegraphics[width=0.7\textwidth]{out/diagrams/signatures/delete/delete_signature}
  \caption{Diagramme de séquence de cas d'utilisation « Supprimer une signature  »}
  \label{fig:sequence_delete_signature}
\end{figure}

\begin{figure}[H]
  \centering
  \includegraphics[width=0.7\textwidth]{out/diagrams/signatures/view/view_signature}
  \caption{Diagramme de séquence de cas d'utilisation « Visualiser une signature  »}
  \label{fig:sequence_view_signature}
\end{figure}

\begin{figure}[H]
  \centering
  \includegraphics[width=0.7\textwidth]{out/diagrams/signatures/update/update_signature}
  \caption{Diagramme de séquence de cas d'utilisation « Modifier une signature  »}
  \label{fig:sequence_update_signature}
\end{figure}

\subsubsection{Analyse détaillée}
La présentation de démarche d'analyse fonctionnelle d'un sprint est très importante pour la satisfaction d'un client parce qu'elle consiste à caractériser les fonctions offertes par un produit.
Donc, nous allons faire l'analyse des différents cas d'utilisation en utilisant le diagramme de classes d'analyse.

Les objets de diagramme d'analyse sont les suivants :
% View
\begin{itemize}
  \item \textbf{View} : représente la vue de l'application.
  \item \textbf{ViewModel} : représente le modèle de la vue.
  \item \textbf{Model} : représente le modèle de l'application.
  \item \textbf{EliseWebService} : représente le web service de l'application Elise.
\end{itemize}

\textbf{•	Diagramme de classe d'analyse de sprint 2 }
\newpage

\begin{figure}
  \centering
  \includegraphics[width=1\textwidth]{dca_sprint2}
  \caption{Diagramme de classe d'analyse de sprint 2}
  \label{fig:class_analyse_signatures}
\end{figure}


\subsubsection{Conception}

Après la présentation des diagrammes d'analyse, nous avons présenté dans cette partie les diagrammes de conception.\\ 
Nous allons présenter dans cette partie les diagrammes de conception de sprint 2. \\
\textbf{•	Diagramme de classe de conception de sprint 2 : « Gestion des signatures »}

% add image
\begin{figure}[H]
  \centering
  \includegraphics[width=1\textwidth]{dcc_spint2}
  \caption{Diagramme de classe de conception de sprint 2 : « Gestion des signatures »}
  \label{fig:class_diagram_signatures}
\end{figure}


\begin{figure}[H]
  \centering
  \includegraphics[width=1\textwidth]{out/diagrams/signatures/sequence_create/sequence_create_signature}
  \caption{Diagramme de séquence de conception de cas d'utilisation « Créer une signature  »}
  \label{fig:sequence_conception_create_signature}
\end{figure}

\begin{figure}[H]
  \centering
  \includegraphics[width=1\textwidth]{out/diagrams/signatures/sequence_delete/sequence_delete_signature}
  \caption{Diagramme de séquence de conception de cas d'utilisation « Supprimer une signature  »}
  \label{fig:sequence_conception_delete_signature}
\end{figure}

\begin{figure}[H]
  \centering
  \includegraphics[width=0.7\textwidth]{out/diagrams/signatures/sequence_view/sequence_view_signature}
  \caption{Diagramme de séquence de conception de cas d'utilisation « Visualiser une signature  »}
  \label{fig:sequence_conception_view_signature}
\end{figure}

\begin{figure}[H]
  \centering
  \includegraphics[width=1\textwidth]{out/diagrams/signatures/sequence_update/sequence_update_signature}
  \caption{Diagramme de séquence de conception de cas d'utilisation « Modifier une signature  »}
  \label{fig:sequence_conception_update_signature}
\end{figure}





\subsubsection{Réalisation}

Après la présentation des diagrammes d'analyse, nous avons présenté dans cette partie des captures d'écran de l'application.

\textbf{•	Interface de création d'une signature:}

Cette capture d'écran, représente l'interface de création d'une signature par un utilisateur

% add image
\begin{figure}[H]
  \centering
  \includegraphics[width=0.35\textwidth, height=0.35\textheight,keepaspectratio=true]{signature_creation}
  \caption{Interface de création d'une signature}
  \label{fig:signature_creation}
\end{figure}

\textbf{•	Signer puis cliquer sur « save » : Interface de saisie du titre:}\\
Cette capture d'écran, représente l'interface de saisie du titre de la signature par un utilisateur
% add image
\begin{figure}[H]
  \centering
  \includegraphics[width=0.35\textwidth, height=0.35\textheight,keepaspectratio=true]{signature_form}
  \caption{Interface de saisie du titre de la signature}
  \label{fig:signature_title}
\end{figure}

\textbf{•	Interface de visualisation d'une signature:}\\
Cette capture d'écran, représente l'interface de visualisation d'une signature par un utilisateur

% add image
\begin{figure}[H]
  \centering
  \includegraphics[width=0.35\textwidth, height=0.35\textheight,keepaspectratio=true]{signature_preview}
  \caption{Interface de visualisation d'une signature}
  \label{fig:signature_view}
\end{figure}

\textbf{•	Interface de suppression d'une signature:}\\
Cette capture d'écran, représente l'interface de suppression d'une signature par un utilisateur

% add image
\begin{figure}[h!]
  \centering
  \includegraphics[width=0.35\textwidth, height=0.35\textheight,keepaspectratio=true]{signature_delete}
  \caption{Interface de suppression d'une signature}
  \label{fig:signature_delete}
\end{figure}
% add image
\begin{figure}[h!]
  \centering
  \includegraphics[width=0.35\textwidth, height=0.35\textheight,keepaspectratio=true]{signature_delete_confirm}
  \caption{Interface de confirmation de suppression d'une signature}
  \label{fig:signature_delete_confirm}
\end{figure}
% add image

\textbf{•	Interface de modification d'une signature:}\\
Cette capture d'écran, représente l'interface de modification d'une signature par un utilisateur
\begin{figure}[h!]
  \centering
  \includegraphics[width=0.35\textwidth, height=0.35\textheight,keepaspectratio=true]{signature_update}
  \caption{Interface de modification d'une signature}
  \label{fig:signature_update}
\end{figure}



\subsection{Sprint Review:}
A la fin de ce sprint, nous avons planifié une réunion dans la société Neoledge afin de vérifier notre démarche de travail par rapport au besoin de client tout en respectant le délai que nous avons prévu.

Nous avons fait une démonstration durant laquelle nous allons présenter notre incrément :

\begin{itemize}
  \item La création d'une signature.
  \item La visualisation de la signature.
  \item La modification de la signature.
  \item La suppression de la signature.
\end{itemize}

\subsection{Sprint retrospective:}

Après la Sprint Review, nous avons réfléchi à des pistes pour améliorer la qualité et l'efficacité de notre application.

\noindent\textbf{•	Ce qui s'est bien passé :}
\begin{itemize}
  \item Nous avons bien partagé les tâches entre nous à travers le logiciel Azure DevOps. 
  \item Nous avons terminé le sprint dans le délai.
\end{itemize}

\noindent\textbf{•	Ce qui s'est mal passé :}
\begin{itemize}
  \item L'intégration du bibliothèque vue-signature-pad
  \item Un problème rencontré lors de la sauvegarde de la signature était que l'image enregistrée occupait la totalité de l'espace du pad, ce qui nécessitait ensuite une étape de découpe pour obtenir uniquement la partie signée.
\end{itemize}

% sprint =3
\section{Sprint 3 « Gestion des documents »}
\subsection{Sprint Goal:}

L'objectif de ce sprint est de développer et mettre en place un système de gestion des documents permettant aux utilisateurs de consulter les informations du document, consulter et gérer un fichier dans un document, consulter et gérer les tâches dans un document, chercher des documents et consulter la liste des documents en favoris

\pagebreak

\subsection{Sprint Backlog « Gestion des documents »}

\begin{adjustwidth}{-1cm}{}
  % \usepackage{longtable}
    
    \begin{longtable}{|c|p{6cm}|c|p{6cm}|c|}
      % \centering
      \hline
      \textbf{ID} & \textbf{User story} & \textbf{ID}  & \textbf{Tâche} & \textbf{Durée} \\
      \hline
      % Chercher des documents
      \multirow{3}{*}{1} & En tant qu'utilisateur, je veux rechercher des documents en utilisant différents critères, tels que le nom, la date, le type ou le contenu, afin de trouver rapidement les documents dont j'ai besoin. & 1.1 & Préparer les interfaces sur Figma. & \multirow{3}{*}{2 Jour} \\
      \cline{3-4}
      & & 1.2 & Développer l'interface de recherche des documents. & \\
      \cline{3-4}
      & & 1.3 & Développer une fonction qui permet de chercher des documents. & \\
      \cline{1-5}
      % Consulter la liste des documents en favoris
      \multirow{3}{*}{2} & En tant qu'utilisateur, je veux afficher les favoris, afin de trouver rapidement les documents dont j'ai besoin. &&& \multirow{3}{*}{2 Jour} \\
      \cline{3-4}
      & & 2.2 & Développer l'interface des favoris. & \\
      \cline{3-4}
      & & 2.3 & Développer une fonction qui permet de récupérer la liste des documents en favoris. & \\
      \cline{1-5}
      % Accéder à un document
      \multirow{3}{*}{3} & En tant qu'utilisateur, je veux récupérer les informations relatives à un document, telles que la date de création, les utilisateurs qui ont accédé au document et les tâches associées, afin de suivre l'historique du document et de mieux comprendre son contexte, les modifier ou de les signer. & 3.1 & Préparer les interfaces sur Figma. & \multirow{3}{*}{2 Jour} \\
      \cline{3-4}
      & & 3.2 & Développer l'interface des documents. & \\
      \cline{3-4}
      & & 3.3 & Développer une fonction qui permet d'accéder à un document. & \\
      \cline{1-5}
      % Télécharger les fichiers d'un document
      \multirow{3}{*}{4} & En tant qu'utilisateur, je veux télécharger les fichiers attachés à un document, afin de garder une copie locale.. & 4.1 & Préparer l'interfaces sur Figma. & \multirow{3}{*}{2 Jour} \\
      \cline{3-4}
      & & 4.2 & Développer l'interface de téléchargement des fichiers. & \\
      \cline{3-4}
      & & 4.3 & Développer une fonction qui permet de télécharger les fichiers d'un document. & \\
      \cline{1-5}
      % Consulter la liste des fichiers d'un document
      \multirow{3}{*}{5} & En tant qu’utilisateur, je veux consulter la liste des fichiers attachés à un document, afin de les visualiser. & 5.1 & Préparer les interfaces sur Figma. & \multirow{3}{*}{2 Jour} \\
      \cline{3-4}
      & & 5.2 & Développer l'interface de liste des fichiers. & \\
      \cline{3-4}
      & & 5.3 & Développer une fonction qui permet de récupérer la liste des fichiers d'un document. & \\
      \cline{1-5}
      % Ajouter un fichier à un document
      \multirow{3}{*}{6} & En tant qu'utilisateur, je veux ajouter un fichier à un document, afin de le partager avec les autres utilisateurs. & 6.1 & Préparer l'interface sur Figma. & \multirow{3}{*}{2 Jour} \\
      \cline{3-4}
      & & 6.2 & Développer l'interface d'ajout des fichiers. & \\
      \cline{3-4}
      & & 6.3 & Développer une fonction qui permet ajouter un fichier à un document. & \\
      \cline{1-5}
      % Consulter la liste des tâches d'un document
      \multirow{3}{*}{7} & En tant qu'utilisateur, je veux consulter la liste des tâches associées à un document, afin de suivre l'état d'avancement des tâches et de savoir qui est responsable de chaque tâche. & 7.1 & Préparer les interfaces sur Figma. & \multirow{3}{*}{2 Jour} \\
      \cline{3-4}
      & & 7.2 & Développer l'interface de liste des tâches. & \\
      \cline{3-4}
      & & 7.3 & Développer une fonction qui permet de récupérer la liste des tâches d'un document. & \\
      \cline{1-5}
      % Ajouter une tâche à un document
      \multirow{3}{*}{8} & En tant qu'utilisateur, je veux demander une tâche dans un document en assignant une personne responsable, afin de faciliter la collaboration et le suivi des tâches. & 8.1 & Préparer l'interface sur Figma. & \multirow{3}{*}{2 Jour} \\
      \cline{3-4}
      & & 8.2 & Développer l'interface d'ajout des tâches. & \\
      \cline{3-4}
      & & 8.3 & Développer une fonction qui permet d'ajouter une tâche à un document. & \\
      \cline{1-5}
      % Transférer une tâche 
      \multirow{3}{*}{9} & En tant qu'utilisateur, je veux transferer une tâche dans un document a un autre utilisateur, afin que ce dernier puisse la prendre en charge. & 9.1 & Préparer l'interface sur Figma. & \multirow{3}{*}{2 Jour} \\
      \cline{3-4}
      & & 9.2 & Développer l'interface de transfert des tâches. & \\
      \cline{3-4}
      & & 9.3 & Développer une fonction qui permet de transférer une tâche dans un document à un autre utilisateur. & \\
      \cline{1-5}
      % traitement d'une tâche
      \multirow{3}{*}{10} &En tant qu'utilisateur, je veux terminer une tâche dans un document, afin de clôturer la tâche et de faciliter le suivi du projet. & 10.1 & Préparer les interfaces sur Figma. & \multirow{3}{*}{2 Jour} \\
      \cline{3-4}
      & & 10.2 & Développer les interfaces de traitement des tâches. & \\
      \cline{3-4}
      & & 10.3 & Développer une fonction qui permet de traiter une tâche dans un document. & \\
      \cline{1-5}
  \hline
  \caption{Sprint backlog du Sprint 3}
  \label{tab:sprint-backlog-3}
\end{longtable}
\end{adjustwidth}



\subsection{Implémentation du Sprint 3}
\textbf{•	Diagramme de cas d'utilisation du sprint 3 : « Gestion des documents »}

% add image
\begin{figure}[H]
  \centering
  \includegraphics[width=0.8\textwidth]{use_case_documents_sprint_3}
  \caption{Diagramme de cas d'utilisation du sprint 3 : « Gestion des documents »}
  \label{fig:UseCaseDiagramSprint3}
\end{figure}


\subsubsection{Analyse des besoins:}
\textbf{•	Description textuelle de cas d'utilisation « Accéder à un document »}

\begin{longtable}{|p{5cm}|p{10cm}|}
\hline
\textbf{Cas d'utilisation}&Accéder à un document\\
\hline
\textbf{Acteurs}&Utilisateur\\
\hline
\textbf{Pré Condition}&Le document existe\\
\hline
\textbf{Post Condition}&Consultation d'un document\\
\hline
\textbf{Scénario Nominal}&
\vspace{-\baselineskip}
\begin{enumerate}
    \setcounter{enumi}{1}
    \item L'utilisateur clique sur le document.
    \item Le système affiche l'interface de document.
    
\end{enumerate}\\
\hline
\textbf{Scénario Alternatif}&
\vspace{-\baselineskip}
\begin{enumerate}
    \setcounter{enumi}{2}
    \item Aucun résultat.
\end{enumerate}\\
\hline
\textbf{Scénario d'exception}&Erreur de connexion\\
\hline
\caption{Description textuelle de cas d'utilisation « Accéder à un document »}
\label{tab:DescriptionTextuelleDeCasDUtilisationAccéderAUnDocument}
\end{longtable}


\textbf{•	Description textuelle de cas d'utilisation « consulter la liste des fichiers d'un document »}

\begin{longtable}{|p{5cm}|p{10cm}|}
\hline
\textbf{Cas d'utilisation}&Consulter la liste des fichiers d'un document\\
\hline
\textbf{Acteurs}&Utilisateur\\
\hline
\textbf{Pré Condition}&Le document existe\\
\hline
\textbf{Post Condition}&Consultation de la liste des fichiers\\
\hline
\textbf{Scénario Nominal}&
\vspace{-\baselineskip}
\begin{enumerate}
    \setcounter{enumi}{1}
    \item L'utilisateur clique sur le bouton fichier.
    \item Le système affiche l'interface de la liste des fichiers.
    
\end{enumerate}\\
\hline
\textbf{Scénario Alternatif}&
\vspace{-\baselineskip}
\begin{enumerate}
    \setcounter{enumi}{2}
    \item Aucun résultat.
\end{enumerate}\\
\hline
\textbf{Scénario d'exception}&Erreur de connexion\\
\hline
\caption{Description textuelle de cas d'utilisation « consulter la liste des fichiers d'un document »}
\label{tab:DescriptionTextuelleDeCasDUtilisationConsulterLaListeDesFichiersDUnDocument}
\end{longtable}

\textbf{•	Description textuelle de cas d'utilisation « ajouter un fichier à un document »}

\begin{longtable}{|p{5cm}|p{10cm}|}
\hline
\textbf{Cas d'utilisation}&Ajouter un fichier à un document\\
\hline
\textbf{Acteurs}&Utilisateur\\
\hline
\textbf{Pré Condition}&Le document existe\\
\hline
\textbf{Post Condition}&Fichier ajouté\\
\hline
\textbf{Scénario Nominal}&
\vspace{-\baselineskip}
\begin{enumerate}
    \setcounter{enumi}{1}
    \item L'utilisateur clique sur le bouton ajouter.
    \item Le système affiche l'interface d'ajout des fichiers.
    \item L'utilisateur ajoute le fichier et clique sur le bouton confirmer.
    \item le système affiche un message des succès.
    
    
\end{enumerate}\\
\hline
\textbf{Scénario Alternatif}&
\vspace{-\baselineskip}
\begin{enumerate}
    \setcounter{enumi}{3}
    \item L'utilisateur annule l'ajout du fichier.
\end{enumerate}\\
\hline
\textbf{Scénario d'exception}&Erreur de connexion\\
\hline
\caption{Description textuelle de cas d'utilisation « ajouter un fichier à un document »}
\label{tab:DescriptionTextuelleDeCasDUtilisationAjouterUnFichierAUnDocument}

\end{longtable}


% \textbf{•	Description textuelle de cas d'utilisation « supprimer un fichier dans un document »}

% \begin{longtable}{|p{5cm}|p{10cm}|}
% \hline
% \textbf{Cas d'utilisation}&Supprimer un fichier d'un document\\
% \hline
% \textbf{Acteurs}&Utilisateur\\
% \hline
% \textbf{Pré Condition}&Le document existe\\
% \hline
% \textbf{Post Condition}&Fichier supprimé\\
% \hline
% \textbf{Scénario Nominal}&
% \vspace{-\baselineskip}
% \begin{enumerate}
%     \setcounter{enumi}{1}
%     \item L'utilisateur clique sur le bouton supprimer.
%     \item Le système affiche une alerte de confirmation.
%     \item L'utilisateur clique sur le bouton confirmer.
%     \item le système supprime le fichier et affiche un message des succès.
    
    
    
% \end{enumerate}\\
% \hline
% \textbf{Scénario Alternatif}&
% \vspace{-\baselineskip}
% \begin{enumerate}
%     \setcounter{enumi}{3}
%     \item L'utilisateur annule la suppression du fichier.
% \end{enumerate}\\
% \hline
% \textbf{Scénario d'exception}&Erreur de connexion\\
% \hline
% \end{longtable}



\textbf{•	Description textuelle de cas d'utilisation « consulter la liste des tâches d'un document »}

\begin{longtable}{|p{5cm}|p{10cm}|}
\hline
\textbf{Cas d'utilisation}&Consulter la liste des tâches d'un document\\
\hline
\textbf{Acteurs}&Utilisateur\\
\hline
\textbf{Pré Condition}&Le document existe\\
\hline
\textbf{Post Condition}&Consultation des tâches\\
\hline
\textbf{Scénario Nominal}&
\vspace{-\baselineskip}
\begin{enumerate}
    \setcounter{enumi}{1}
    \item L'utilisateur clique sur le bouton tâches.
    \item Le système affiche l'interface de la liste des tâches.
    
\end{enumerate}\\
\hline
\textbf{Scénario Alternatif}&
\vspace{-\baselineskip}
\begin{enumerate}
    \setcounter{enumi}{3}
    \item Aucun tâche trouvée.
\end{enumerate}\\
\hline
\textbf{Scénario d'exception}&Erreur de connexion\\
\hline
\caption{Description textuelle de cas d'utilisation « consulter la liste des tâches d'un document »}
\label{tab:DescriptionTextuelleDeCasDUtilisationConsulterLaListeDesTachesDUnDocument}
\end{longtable}



\textbf{•	Description textuelle de cas d'utilisation « Demander une tâche »}

\begin{longtable}{|p{5cm}|p{10cm}|}
\hline
\textbf{Cas d'utilisation}&Demander une tâche\\
\hline
\textbf{Acteurs}&Utilisateur\\
\hline
\textbf{Pré Condition}&La document a au moins une tâche\\
\hline
\textbf{Post Condition}&Tâche demandée\\
\hline
\textbf{Scénario Nominal}&
\vspace{-\baselineskip}
\begin{enumerate}
    \setcounter{enumi}{1}
    \item L'utilisateur clique sur le bouton demander.
    \item Le système affiche un modal de demande de tâche.
    \item L'utilistauer rempli le formulaire.
    \item L'utilisateur clique sur le bouton confirmer.
    \item Le système cache le modal de demande de tâche.
    \item Le système affiche un message de succès.
\end{enumerate}\\
\hline
\textbf{Scénario Alternatif}&
\vspace{-\baselineskip}
\begin{enumerate}
    \setcounter{enumi}{4}
    \item L'utilisateur annule la demande de tâche.
    \item Le système cache le modal de demande de tâche.
\end{enumerate}\\
\hline
\textbf{Scénario d'exception}&Erreur de connexion\\
\hline
\caption{Description textuelle de cas d'utilisation « Demander une tâche »}
\label{tab:DescriptionTextuelleDeCasDUtilisationDemanderUneTache}
\end{longtable}


\textbf{•	Description textuelle de cas d'utilisation « Tranférer une tâche »}

\begin{longtable}{|p{5cm}|p{10cm}|}
\hline
\textbf{Cas d'utilisation}&Tranférer une tâche\\
\hline
\textbf{Acteurs}&Utilisateur\\
\hline
\textbf{Pré Condition}&La document a au moins une tâche\\
\hline
\textbf{Post Condition}&Tâche transférée\\
\hline
\textbf{Scénario Nominal}&
\vspace{-\baselineskip}
\begin{enumerate}
    \setcounter{enumi}{1}
    \item L'utilisateur clique sur le bouton transférer.
    \item Le système affiche un modal de transfert de tâche.
    \item L'utilistauer rempli le formulaire.
    \item L'utilisateur clique sur le bouton confirmer.
    \item Le système cache le modal de transfert de tâche.
    \item Le système affiche un message de succès.
\end{enumerate}\\
\hline
\textbf{Scénario Alternatif}&
\vspace{-\baselineskip}
\begin{enumerate}
    \setcounter{enumi}{4}
    \item L'utilisateur annule la transfert de tâche.
    \item Le système cache le modal de transfert de tâche.
\end{enumerate}\\
\hline
\textbf{Scénario d'exception}&Erreur de connexion\\
\hline
\caption{Description textuelle de cas d'utilisation « Tranférer une tâche »}
\label{tab:DescriptionTextuelleDeCasDUtilisationTranfererUneTache}
\end{longtable}

\textbf{•	Description textuelle de cas d'utilisation « Traiter une tâche »}

\begin{longtable}{|p{5cm}|p{10cm}|}
\hline
\textbf{Cas d'utilisation}&Traiter une tâche\\
\hline
\textbf{Acteurs}&Utilisateur\\
\hline
\textbf{Pré Condition}&La document a au moins une tâche\\
\hline
\textbf{Post Condition}&Tâche terminée\\
\hline
\textbf{Scénario Nominal}&
\vspace{-\baselineskip}
\begin{enumerate}
    \setcounter{enumi}{1}
    \item L'utilisateur clique sur le bouton traiter.
    \item Le système affiche un modal de confirmation de traitement de tâche.
    \item L'utilisateur clique sur le bouton confirmer.
    \item Le système cache le modal de confirmation de traitement de tâche.
    \item Le système affiche un message de succès.
\end{enumerate}\\
\hline
\textbf{Scénario Alternatif}&
\vspace{-\baselineskip}
\begin{enumerate}
    \setcounter{enumi}{4}
    \item L'utilisateur annule la traitement de tâche.
    \item Le système cache le modal de confirmation de traitement de tâche.
\end{enumerate}\\
\hline
\textbf{Scénario d'exception}&Erreur de connexion\\
\hline
\caption{Description textuelle de cas d'utilisation « Traiter une tâche »}
\label{tab:DescriptionTextuelleDeCasDUtilisationTraiterUneTache}
\end{longtable}



\textbf{•	Description textuelle de cas d'utilisation « Consulter la liste des documents en favoris  »}

\begin{longtable}{|p{5cm}|p{10cm}|}
\hline
\textbf{Cas d'utilisation}&Consulter la liste des documents en favoris\\
\hline
\textbf{Acteurs}&Utilisateur\\
\hline
\textbf{Pré Condition}&L'utilisateur a au moins un document en favoris\\
\hline
\textbf{Post Condition}&Affichage de la liste des documents en favoris\\
\hline
\textbf{Scénario Nominal}&
\vspace{-\baselineskip}
\begin{enumerate}
    \setcounter{enumi}{1}
    \item L'utilisateur clique sur le bouton favoris.
    \item Le système affiche la liste des documents en favoris.
\end{enumerate}\\
\hline
\textbf{Scénario Alternatif}&
\vspace{-\baselineskip}
\begin{enumerate}
    \setcounter{enumi}{2}
    \item Aucun document en favoris.
\end{enumerate}\\
\hline
\textbf{Scénario d'exception}&Erreur de connexion\\
\hline
\caption{Description textuelle de cas d'utilisation « Consulter la liste des documents en favoris  »}
\label{tab:DescriptionTextuelleDeCasDUtilisationConsulterLaListeDesDocumentsEnFavoris}
\end{longtable}




\textbf{•	Description textuelle de cas d'utilisation « Chercher des documents »}

\begin{longtable}{|p{5cm}|p{10cm}|}
\hline
\textbf{Cas d'utilisation}&Chercher des documents\\
\hline
\textbf{Acteurs}&Utilisateur\\
\hline
\textbf{Pré Condition}&\\
\hline
\textbf{Post Condition}&Liste des documents correspondant à la recherche\\
\hline
\textbf{Scénario Nominal}&
\vspace{-\baselineskip}
\begin{enumerate}
    \setcounter{enumi}{1}
    \item L'utilisateur clique sur le bouton rechercher.
    \item Le système affiche l'inteface de recherche des documents.
    \item L'utilisateur rempli le formulaire de recherche.
    \item Le système affiche la liste des documents correspondant à la recherche.
\end{enumerate}\\
\hline
\textbf{Scénario Alternatif}&
\vspace{-\baselineskip}
\begin{enumerate}
    \setcounter{enumi}{4}
    \item Aucun document correspondant à la recherche.
\end{enumerate}\\
\hline
\textbf{Scénario d'exception}&Erreur de connexion\\
\hline
\caption{Description textuelle de cas d'utilisation « Chercher des documents »}
\label{tab:DescriptionTextuelleDeCasDUtilisationChercherDesDocuments}
\end{longtable}

\begin{figure}[H]
  \centering
  \includegraphics[width=1\textwidth]{design_show_documents}
  \caption{Maquette de l'affichage des documents}
  \label{fig:design_show_documents}
\end{figure}

\begin{figure}[H]
  \centering
  \includegraphics[width=1\textwidth]{design_content}
  \caption{Maquette de l'affichage de contenu d'un document}
  \label{fig:design_show_document}
\end{figure}

\begin{figure}[H]
  \centering
  \includegraphics[width=1\textwidth]{design_task_managment}
  \caption{Maquette de l'affichage de gestion des tâches}
  \label{fig:design_manage_tasks}
\end{figure}

% Add sequence system
\begin{figure}[H]
  \centering
  \includegraphics[width=0.7\textwidth]{out/diagrams/documents/preview/preview_document}
  \caption{Diagramme de séquence de cas d'utilisation « Accéder à un document  »}
  \label{fig:sequence_Accederaundocument}
\end{figure}
\begin{figure}[H]
  \centering
  \includegraphics[width=0.7\textwidth]{out/diagrams/documents/previewFiles/preview_files_document}
  \caption{Diagramme de séquence de cas d'utilisation « Consulter la liste des fichiers d'un document  »}
  \label{fig:sequence_previewFiles}
\end{figure}
\begin{figure}[H]
  \centering
  \includegraphics[width=0.7\textwidth]{out/diagrams/documents/previewTasks/preview_tasks_document}
  \caption{Diagramme de séquence de cas d'utilisation « Consulter la liste des tâches d'un document  »}
  \label{fig:sequence_previewTasks}
\end{figure}
\begin{figure}[H]
  \centering
  \includegraphics[width=1\textwidth]{out/diagrams/documents/transfer_task/transfer_task}
  \caption{Diagramme de séquence de cas d'utilisation « Transferer une tâche  »}
  \label{fig:sequence_transfer_task}
\end{figure}
\begin{figure}[H]
  \centering
  \includegraphics[width=1\textwidth]{out/diagrams/documents/traiter_task/traiter_task}
  \caption{Diagramme de séquence de cas d'utilisation « Traiter une tâche  »}
  \label{fig:sequence_traiter_task}
\end{figure}
\begin{figure}[H]
  \centering
  \includegraphics[width=1\textwidth]{out/diagrams/documents/add_file/add_file}
  \caption{Diagramme de séquence de cas d'utilisation « Ajoutez un fichier à un document  »}
  \label{fig:sequence_add_file}
\end{figure}
\begin{figure}[H]
  \centering
  \includegraphics[width=0.7\textwidth]{out/diagrams/documents/favoris/favorit_document}
  \caption{Diagramme de séquence de cas d'utilisation « Consulter la liste des documents en favoris  »}
  \label{fig:sequence_favorit_document}
\end{figure}
\begin{figure}[H]
  \centering
  \includegraphics[width=0.7\textwidth]{out/diagrams/documents/chercher/charcher_document}
  \caption{Diagramme de séquence de cas d'utilisation « Chercher des documents  »}
  \label{fig:sequence_charcher_document}
\end{figure}

\subsubsection{Analyse détaillée}
\textbf{•	Diagramme de classe d'analyse de sprint "3" }
\newpage

\begin{figure}
  \centering
  \includegraphics[width=1\textwidth]{dca_sprint3}
  \caption{Diagramme de classe d'analyse de sprint 3}
  \label{fig:class_analyse_signatures3}
\end{figure}


\subsubsection{Conception}

Après la présentation des diagrammes d'analyse, nous avons présenté dans cette partie les diagrammes de conception.
Nous allons présenter dans cette partie les diagrammes de conception de sprint 3.
\newpage 
\begin{landscape}

\textbf{•	Diagramme de classe de conception de sprint 3 : « Gestion des documents »}

\begin{figure}[H]
  \centering
  \includegraphics[height=0.8\textheight]{dcc_spint3}
  \caption{Diagramme de classe de conception de sprint 3 : « Gestion des signatures »}
  \label{fig:class_diagram_signatures3}
\end{figure}
\end{landscape}
\newpage
\begin{figure}[H]
  \centering
  \includegraphics[width=1\textwidth]{out/diagrams/documents/sequence_preview/sequence_preview}
  \caption{Diagramme de séquence de conception de cas d'utilisation « Accéder à un document »}
  \label{fig:sequence_conception_preview_document}
\end{figure}
\begin{figure}[H]
  \centering
  \includegraphics[width=1\textwidth]{out/diagrams/documents/sequence_preview_files/sequence_preview_files}
  \caption{Diagramme de séquence de conception de cas d'utilisation « Consulter la liste des fichiers d'un document »}
  \label{fig:sequence_conception_previewFiles}
\end{figure}
\begin{figure}[H]
  \centering
  \includegraphics[width=1\textwidth]{out/diagrams/documents/sequence_preview_tasks/sequence_preview_tasks}
  \caption{Diagramme de séquence de conception de cas d'utilisation « Consulter la liste des tâches d'un document »}
  \label{fig:sequence_conception_previewTasks}
\end{figure}
\begin{figure}[H]
  \centering
  \includegraphics[width=1\textwidth]{out/diagrams/documents/sequence_favoris/sequence_favoris}
  \caption{Diagramme de séquence de conception de cas d'utilisation « Consulter la liste des documents en favoris »}
  \label{fig:sequence_conception_favoritDocument}
\end{figure}
\begin{figure}[H]
  \centering
  \includegraphics[width=1\textwidth]{out/diagrams/documents/sequence_transfer_task/sequence_transfer_task}
  \caption{Diagramme de séquence de conception de cas d'utilisation « Transferer une tâche »}
  \label{fig:sequence_conception_transferTask}
\end{figure}
\begin{figure}[H]
  \centering
  \includegraphics[width=1\textwidth]{out/diagrams/documents/sequence_traiter_task/sequence_traiter_task}
  \caption{Diagramme de séquence de conception de cas d'utilisation « Traiter une tâche »}
  \label{fig:sequence_conception_traiterTask}
\end{figure}
\begin{figure}[H]
  \centering
  \includegraphics[width=1\textwidth]{out/diagrams/documents/sequence_add_file/sequence_add_file}
  \caption{Diagramme de séquence de conception de cas d'utilisation « Ajoutez un fichier à un document »}
  \label{fig:sequence_conception_addFile}
\end{figure}
\begin{figure}[H]
  \centering
  \includegraphics[width=1\textwidth]{out/diagrams/documents/sequence_chercher/sequence_chercher}
  \caption{Diagramme de séquence de conception de cas d'utilisation « Chercher des documents »}
  \label{fig:sequence_conception_charcherDocument}
\end{figure}

\subsubsection{Réalisation}

Après la présentation des diagrammes d'analyse, nous avons présenté dans cette partie des captures d'écran de l'application.

\textbf{•	Accès aux documents:}

Cette capture d'écran, représente l'interface d'accès aux documents par un utilisateur

% add image
\begin{figure}[H]
  \centering
  \includegraphics[width=0.35\textwidth, height=0.35\textheight,keepaspectratio=true]{acces_aux_document}
  \caption{Interface d'accès aux documents}
  \label{fig:acces_aux_document}
\end{figure}

\textbf{•	Consulter les fichiers d'un document:}

Cette capture d'écran, représente l'interface de consultation des fichiers d'un document par un utilisateur

% add image
\begin{figure}[H]
  \centering
  \includegraphics[width=0.35\textwidth, height=0.35\textheight,keepaspectratio=true]{consult_document_files}
  \caption{Interface de consultation des fichiers d'un document}
  \label{fig:consult_document_files}
\end{figure}

\textbf{•	Consulter les tâches d'un document:}

Cette capture d'écran, représente l'interface de consultation des tâches d'un document par un utilisateur

% add image
\begin{figure}[H]
  \centering
  \includegraphics[width=0.35\textwidth, height=0.35\textheight,keepaspectratio=true]{consult_document_tasks}
  \caption{Interface de consultation des tâches d'un document}
  \label{fig:consult_document_tasks}
\end{figure}

\textbf{•	Demander une tâche:}

Cette capture d'écran, représente l'interface de demande d'une tâche par un utilisateur

% add image
\begin{figure}[H]
  \centering
  \includegraphics[width=0.35\textwidth, height=0.35\textheight,keepaspectratio=true]{ask_task}
  \caption{Interface de demande d'une tâche}
  \label{fig:ask_task}
\end{figure}

\textbf{•	Transférer une tâche:}

Cette capture d'écran, représente l'interface de transfert d'une tâche par un utilisateur

% add image
\begin{figure}[H]
  \centering
  \includegraphics[width=0.35\textwidth, height=0.35\textheight,keepaspectratio=true]{transfer_task}
  \caption{Interface de transfert d'une tâche}
  \label{fig:transfer_task}
\end{figure}

\textbf{•	Traitement d'une tâche dans un document:}

Cette capture d'écran, représente l'interface de traitement d'une tâche dans un document par un utilisateur

% add image
\begin{figure}[H]
  \centering
  \includegraphics[width=0.35\textwidth, height=0.35\textheight,keepaspectratio=true]{end_task}
  \caption{Interface de traitement d'une tâche dans un document}
  \label{fig:end_task}
\end{figure}




\textbf{•	Ajout de fichiers à un document:}

Cette capture d'écran, représente l'interface d'ajout de fichiers à un document par un utilisateur

% add image
\begin{figure}[H]
  \centering
  \includegraphics[width=0.35\textwidth, height=0.35\textheight,keepaspectratio=true]{add_document_files}
  \caption{Interface d'ajout de fichiers à un document}
  \label{fig:add_document_files}
\end{figure}

% \textbf{• Suppression de fichiers d'un document:}

% Cette capture d'écran, représente l'interface de suppression de fichiers d'un document par un utilisateur

% % add image
% \begin{figure}[H]
%   \centering
%   \includegraphics[width=0.35\textwidth, height=0.35\textheight,keepaspectratio=true]{delete_document_files}
%   \caption{Interface de suppression de fichiers d'un document}
%   \label{fig:delete_document_files}
% \end{figure}


\textbf{•	Recherche d'un document:}

Cette capture d'écran, représente l'interface de recherche d'un document par un utilisateur

% add image
\begin{figure}[H]
  \centering
  \includegraphics[width=0.35\textwidth, height=0.35\textheight,keepaspectratio=true]{search_document}
  \caption{Interface de recherche d'un document}
  \label{fig:search_document}
\end{figure}

\textbf{•	Consulter les documents en favoris:}

Cette capture d'écran, représente l'interface de consultation des documents en favoris par un utilisateur

% add image
\begin{figure}[H]
  \centering
  \includegraphics[width=0.35\textwidth, height=0.35\textheight,keepaspectratio=true]{consult_favoris_documents}
  \caption{Interface de consultation des documents en favoris}
  \label{fig:consult_favoris_documents}
\end{figure}



\subsection{Sprint Review:}
A la fin de ce sprint, nous avons planifié une réunion dans la société Neoledge avec le vise avis à Lille, France afin de vérifier notre démarche de travail par rapport au besoin de client tout en respectant le délai que nous avons prévu.
Nous avons fait une démonstration durant laquelle nous allons présenter notre incrément :
\begin{itemize}
  \item La consultation d'un document.
  \item La gestion des fichiers d'un document.
  \item La gestion des taches d'un document.
  \item La recherche d'un document.
  \item La consultation de la liste des documents en favoris.
\end{itemize}

\subsection{Sprint Retrospective:}

Après la Sprint Review, nous avons réfléchi à des pistes pour améliorer la qualité et l'efficacité de notre application.
\noindent\textbf{•	Ce qui a bien passé :}
Nous avons terminé le sprint dans le délai.
\noindent\textbf{•	Ce qui s'est mal passé :}
Manque de documentation de l'ionic 


% SPrint 4 visualisation et signature d'un fichier
\section{Sprint 4 (Visualisation et signature d'un fichier)}
\subsection{Sprint Goal:}
L'objectif de ce sprint est de développer et mettre en place un système de visualisation et signature d'un fichier permettant aux utilisateurs de consulter le fichier et le signer par plusieurs méthodes.


\subsection{Sprint Backlog « Visualisation et signature d'un fichier »:}

% \begin{longtable}{|p{4cm}|p{7cm}|p{2cm}|p{2cm}|}
%   \hline
%   \textbf{Les items} &\textbf{Les tâches} & \textbf{Période} & \textbf{Sprint} \\
%   \hline
%   \vspace{-\baselineskip}
%   \begin{enumerate}
%     \setcounter{enumi}{1}
%     \itemsep0em 
%       \item Visualiser un fichier
%       \item Télécharger un fichier
%       \item Signer à main un fichier
%       \item Signé par glisser et déposer une image dans un fichier.
%       \item Modifier la position d'une signature d'un fichier.
%       \item Supprimer une signature dans fichier.
%       \item Confirmer ou annuler la signature d'un fichier
%   \end{enumerate}
%   &
%   \vspace{-\baselineskip}
%   \begin{itemize}
%     \itemsep0em 
%     \item Préparer les interfaces sur Figma.
%     \item Développer l'interface de visualisation d'un fichier.
%     \item Développer un package neo-pdf-viewer qui permet d'afficher un fichier PDF à partir d'une base 64
%     \item Publier le package.
%     \item Intégrer le package dans notre projet.
%     \item Développer une fonction qui permet de télécharger un fichier.
%     \item Développer une fonction qui permet de signer un main un fichier
%     \item Développer le panel qui contient la liste des signatures
%     \item Développer une fonction qui permet de signer par glisser et déposer une image dans un fichier.
%     \item Développer une fonction qui permet de modifier la position d'une signature d'un fichier
%     \item Développer une fonction qui permet de supprimer une signature d'un fichier.
%     \item Développer une fonction qui permet de confirmer ou annuler la signature d'un fichier


%   \end{itemize}
%   &
%   De 8 à 15 février  
%   &
%   4
%   \\
%   \hline
%   \caption{Product Backlog Sprint 4}
%   \label{tab:product_backlog_sprint_4}

% \end{longtable}

\begin{adjustwidth}{-1cm}{}
  % \usepackage{longtable}
    
    \begin{longtable}{|c|p{6cm}|c|p{6cm}|c|}
      % \centering
      \hline
      \textbf{ID} & \textbf{User story} & \textbf{ID}  & \textbf{Tâche} & \textbf{Durée} \\
      \hline
      % Visualiser un fichier
      % Télécharger un fichier
      % Signer à main un fichier
      % Signé par glisser et déposer une image dans un fichier.
      % Modifier la position d'une signature d'un fichier.
      % Supprimer une signature dans fichier.
      % Confirmer ou annuler la signature d'un fichier

      \multirow{3}{*}{1} & En tant qu'utilisateur, je veux
      visualiser des fichiers stockés dans le document, afin de consulter leur contenu sans avoir besoin d'une autre application ou d'un
      autre dispositif. & 1.1 & Préparer les interfaces sur Figma. & \multirow{3}{*}{2 Jour} \\
      \cline{3-4}
      & & 1.2 & Développer l'interface de visualisation d'un fichier. & \\
      \cline{3-4}
      & & 1.3 & Développer un package neo-pdf-viewer qui permet d'afficher un fichier PDF à partir d'une base 64. & \\
      \cline{1-5}

      \multirow{3}{*}{2} & En tant qu'utilisateur, je veux télécharger un fichier, afin d'avoir une copie locale du fichier sur mon appareil. & 2.1 & Préparer les interfaces sur Figma. & \multirow{3}{*}{2 Jour} \\
      \cline{3-4}
      & & 2.2 & Développer une fonction qui permet de télécharger un fichier. & \\
      \cline{1-5}

      \multirow{3}{*}{3} & En tant qu'utilisateur, je souhaite signer des fichiers avec image, Je veux être en mesure de sélectionner une signature pré-enregistrée parmi celles que j'ai créées précédemment, en utilisant simplement un système de glisser déposer pour l'appliquer au document à signer.& 3.1 & Préparer les interfaces sur Figma. & \multirow{3}{*}{2 Jour} \\
      \cline{3-4}
      & & 3.2 & Développer l'interface pour afficher la liste des signatures. & \\
      & & 3.3 & Développer une fonction qui permet d'afficher la liste des signatures. & \\
      & & 3.4 & Développer une fonction qui permet de signer par glisser et déposer une image dans un fichier. & \\
      \cline{1-5}

      \multirow{3}{*}{4} & En tant qu’utilisateur, je souhaite signer des fichiers a main. Je veux signer le document directement en utilisant une méthode de signature manuscrite, en dessinant ma signature. & 4.1 & Préparer les interfaces sur Figma. & \multirow{3}{*}{2 Jour} \\
      \cline{3-4}
      & & 4.2 & Développer une fonction qui permet de signer à main un fichier. & \\
      \cline{1-5}

      \multirow{3}{*}{5} & En tant qu'utilisateur, je souhaite déplacer les signatures non enregistrées, afin de les positionner correctement.& 5.1 & Préparer les interfaces sur Figma. & \multirow{3}{*}{2 Jour} \\
      \cline{3-4}
      & & 5.2 & Développer une fonction qui permet de modifier la position d'une signature d'un fichier. & \\
      \cline{1-5}

      \multirow{3}{*}{6} & En tant qu'utilisateur, je souhaite supprimer les signatures non enregistrées, afin d'annuler la signature.& 6.1 & Préparer l'interface sur figma. & \multirow{3}{*}{2 Jour} \\
      \cline{3-4}
      & & 6.2 & Développer l'interface de suppression d'une signature. & \\
      & & 6.3 & Développer une fonction qui permet de supprimer une signature d'un fichier. & \\
      \cline{1-5}

      \multirow{3}{*}{7} & En tant qu'utilisateur, je souhaite confirmer ou annuler la signature, afin de la valider ou de la supprimer.& 7.1 & Préparer l'interface sur figma. & \multirow{3}{*}{2 Jour} \\
      \cline{3-4}
      & & 7.2 & Développer l'interface de confirmation ou annulation de la signature. & \\
      & & 7.3 & Développer une fonction qui permet de confirmer ou annuler la signature d'un fichier. & \\
      \cline{1-5}
      
  \hline
  \caption{Sprint backlog du Sprint 4}
  \label{tab:sprint-backlog-3}
\end{longtable}
\end{adjustwidth}

\subsection{Implémentation du Sprint 4}
\textbf{•	Diagramme de cas d'utilisation du sprint 4 : « Visualisation et signature d'un fichier »}

% add image
\begin{figure}[H]
  \centering
  \includegraphics[width=0.8\textwidth]{use_case_documents_sprint_4}
  \caption{Diagramme de cas d'utilisation du sprint 4 : « Visualisation et signature d'un fichier »}
  \label{fig:UseCaseDiagram4}
\end{figure}
\subsubsection{Analyse des besoins:}
\textbf{•	Description textuelle de cas d'utilisation « Visualiser un fichier  »}
\begin{longtable}{|p{5cm}|p{10cm}|}
\hline
\textbf{Cas d'utilisation}&Visualiser un fichier\\
\hline
\textbf{Acteurs}&Utilisateur\\
\hline
\textbf{Pré Condition}&Le fichier existe\\
\hline
\textbf{Post Condition}&Le fichier est visualisé\\
\hline
\textbf{Scénario Nominal}&
\vspace{-\baselineskip}
\begin{enumerate}
    \setcounter{enumi}{1}
  \item L'utilisateur clique sur le fichier.
  \item le système affiche le fichier.
\end{enumerate}\\
\hline
\textbf{Scénario Alternatif}&
\vspace{-\baselineskip}
\begin{enumerate}
    \setcounter{enumi}{2}
    \item Aucun résultat.
\end{enumerate}\\
\hline
\textbf{Scénario d'exception}&Erreur de connexion\\
\hline
\caption{Description textuelle de cas d'utilisation « Visualiser un fichier  »}
\label{tab:description-textuelle-de-cas-d-utilisation-visualiser-un-fichier}

\end{longtable}

\textbf{•	Description textuelle de cas d'utilisation « Télécharger un fichier »}
\begin{longtable}{|p{5cm}|p{10cm}|}
\hline
\textbf{Cas d'utilisation}&Télécharger un fichier\\
\hline
\textbf{Acteurs}&Utilisateur\\
\hline
\textbf{Pré Condition}&L'utilisateur est en cours de visualisation d'un fichier\\
\hline
\textbf{Post Condition}&Le fichier est téléchargé\\
\hline
\textbf{Scénario Nominal}&
\vspace{-\baselineskip}
\begin{enumerate}
    \setcounter{enumi}{1}
  \item L'utilisateur clique sur afficher les actions.
  \item Le système affiche les actions.
  \item L'utilisateur clique sur télécharger.
  \item Le système télécharge le fichier.
  \item Le système affiche un message de succès.
\end{enumerate}\\
\hline
\textbf{Scénario Alternatif}&
\vspace{-\baselineskip}
\begin{enumerate}
  \item [4.1] Le système n'arrive pas à télécharger le fichier.
  \item [4.2] Le système affiche un message d'erreur.
  % second alternative
  \item [4.1] La fichier a des signatures non enregistrées.
  \item [4.2] Le système demande si l'utilisateur veut télécharger le fichier sans les signatures.
  \item [4.3.1] L'utilisateur clique sur oui.
  \item [4.3.2] Le système continue de la point 4 du scénario nominal.
  \item [4.3.1] L'utilisateur clique sur non.
  \item [4.3.2] Le système annule le téléchargement.

\end{enumerate}\\
\hline
\textbf{Scénario d'exception}&
\vspace{-\baselineskip}
\begin{enumerate}
  \item [3.1] le système n'a pas le droit de télécharger le fichier.
  \item [3.2] Le système affiche un message d'erreur.
\end{enumerate}\\
\hline
\caption{Description textuelle de cas d'utilisation « Télécharger un fichier »}
\label{tab:description-textuelle-de-cas-d-utilisation-telecharger-un-fichier}
\end{longtable}


\textbf{•	Description textuelle de cas d'utilisation « Signer par glisser et déposer une image dans un fichier »}

\begin{longtable}{|p{5cm}|p{10cm}|}
\hline
\textbf{Cas d'utilisation}&Signer par glisser et déposer une image dans un fichier\\
\hline
\textbf{Acteurs}&Utilisateur\\
\hline
\textbf{Pré Condition}&Le fichier existe\\
\hline
\textbf{Post Condition}&La signature est placée\\
\hline
\textbf{Scénario Nominal}&
\vspace{-\baselineskip}
\begin{enumerate}
    \setcounter{enumi}{1}
    \item L'utilisateur glisse ou clique sur le bouton du panel.
    \item Le système affiche le panel qui contient la liste des signatures.
    \item L'utilisateur glisse une signature.
    \item L'utilisateur dépose la signature dans le fichier.
    \item Le système place la signature dans le fichier.
\end{enumerate}\\
\hline
\textbf{Scénario Alternatif}&
\vspace{-\baselineskip}
\begin{enumerate}
    \item [4.1] L'utilisateur dépose la signature hors du fichier.
    \item [4.2] Le système ne place pas la signature dans le fichier.
\end{enumerate}\\
\hline
\caption{Description textuelle de cas d'utilisation « Signer par glisser et déposer une image dans un fichier »}
\label{tab:description-textuelle-de-cas-d-utilisation-signer-par-glisser-et-deposer-une-image-dans-un-fichier}
\end{longtable}

\textbf{•	Description textuelle de cas d'utilisation « Signer à main un fichier »}

\begin{longtable}{|p{5cm}|p{10cm}|}
\hline
\textbf{Cas d'utilisation}&Signer à main un fichier\\
\hline
\textbf{Acteurs}&Utilisateur\\
\hline
\textbf{Pré Condition}&Le fichier existe\\
\hline
\textbf{Post Condition}&La signature est tracée\\
\hline
\textbf{Scénario Nominal}&
\vspace{-\baselineskip}
\begin{enumerate}
    \setcounter{enumi}{1}
  \item L'utilisateur fait une longue clique sur le fichier
  \item L'utilisateur clique sur afficher les actions.
  \item Le système affiche les actions.
  \item L'utilisateur clique sur le bouton signature à main.
  \item L'utilisateur trace sa signature.
  \item Le système affiche les 2 boutons confirmer et annuler
  \item L'utilisateur clique sur le bouton confirmer
  \item Le système place la signature dans le fichier.
\end{enumerate}\\
\hline
\textbf{Scénario Alternatif}&
\vspace{-\baselineskip}
\begin{enumerate}
    \item [6.1]L'utilisateur clique sur le bouton annuler
    \item [6.2]Le tracé de la signature est effacé.
\end{enumerate}\\
\hline
\caption{Description textuelle de cas d'utilisation « Signer à main un fichier »}
\label{tab:description-textuelle-de-cas-d-utilisation-signer-a-main-un-fichier}
\end{longtable}


\textbf{•	Description textuelle de cas d'utilisation « Modifier la position d'une signature d'un fichier »}

\begin{longtable}{|p{5cm}|p{10cm}|}
\hline
\textbf{Cas d'utilisation}&Modifier la position d'une signature d'un fichier\\
\hline
\textbf{Acteurs}&Utilisateur\\
\hline
\textbf{Pré Condition}&La signature est placée et l'utilisateur n'a pas encore confirmer les changements\\
\hline
\textbf{Post Condition}&La position de la signature est modifiée\\
\hline
\textbf{Scénario Nominal}&
\vspace{-\baselineskip}
\begin{enumerate}
    \setcounter{enumi}{1}
    \item L'utilisateur clique sur la signature.
    \item L'utilisateur deplace et dépose la signature.
    \item Le système modifie la position de la signature.
\end{enumerate}\\
\hline
\textbf{Scénario Alternatif}&
\vspace{-\baselineskip}
\begin{enumerate}
    \item [2.1]L'utilisateur dépose la signature hors du fichier.
    \item [2.1]Le système annule les changements.
\end{enumerate}\\
\hline
\caption{Description textuelle de cas d'utilisation « Modifier la position d'une signature d'un fichier »}
\label{tab:description-textuelle-de-cas-d-utilisation-modifier-la-position-d-une-signature-d-un-fichier}
\end{longtable}

\textbf{•	Description textuelle de cas d'utilisation « Supprimer une signature d'un fichier  »}

\begin{longtable}{|p{5cm}|p{10cm}|}
\hline
\textbf{Cas d'utilisation}&Supprimer une signature d'un fichier \\
\hline
\textbf{Acteurs}&Utilisateur\\
\hline
\textbf{Pré Condition}&La signature est placée et l'utilisateur n'a pas encore confirmer les changements\\
\hline
\textbf{Post Condition}&La signature est supprimée\\
\hline
\textbf{Scénario Nominal}&
\vspace{-\baselineskip}
\begin{enumerate}
    \setcounter{enumi}{1}
    \item L'utilisateur déplace la signature dans la zone de suppression.
    \item Le système supprime la signature d'un fichier
\end{enumerate}\\
\hline
\textbf{Scénario Alternatif}&
\vspace{-\baselineskip}
\begin{enumerate}
    \item [1.1]L'utilisateur dépose la signature hors du fichier.
    \item [1.2]Le système annule les changements.
\end{enumerate}\\
\hline
\caption{Description textuelle de cas d'utilisation « Supprimer une signature d'un fichier  »}
\label{tab:description-textuelle-de-cas-d-utilisation-supprimer-une-signature-d-un-fichier}
\end{longtable}


\textbf{•	Description textuelle de cas d'utilisation « Confirmer ou annuler les changements d'un fichier »}

\begin{longtable}{|p{5cm}|p{10cm}|}
\hline
\textbf{Cas d'utilisation}&Confirmer ou annuler les changements d'un fichier\\
\hline
\textbf{Acteurs}&Utilisateur\\
\hline
\textbf{Pré Condition}&Le fichier est modifié\\
\hline
\textbf{Post Condition}&Les changements sont enregistrés ou annulés\\
\hline
\textbf{Scénario Nominal}&
\vspace{-\baselineskip}
\begin{enumerate}
    \setcounter{enumi}{1}
    \item L'utilisateur clique sur afficher les actions.
    \item Le système affiche les actions.
    \item L'utilisateur clique sur le bouton confirmer.
    \item Le système enregistre les changements et affiche un message des succès.
\end{enumerate}\\
\hline
\textbf{Scénario Alternatif}&
\vspace{-\baselineskip}
\begin{enumerate}
    \item [3.1]L'utilisateur clique sur le bouton annuler.
    \item [3.2]Le système annule les changements.
\end{enumerate}\\
\hline
\textbf{Scénario d'exception}&Erreur de connexion\\
\hline
\caption{Description textuelle de cas d'utilisation « Confirmer ou annuler les changements d'un fichier »}
\label{tab:description-textuelle-de-cas-d-utilisation-confirmer-ou-annuler-les-changements-d-un-fichier}
\end{longtable}


\begin{figure}[H]
  \centering
  \includegraphics[width=0.7\textwidth]{out/diagrams/sprint4/view_file/view_file}
  \caption{Diagramme de séquence de cas d'utilisation « Visualiser un fichier   »}
  \label{fig:sequence_view_file}
\end{figure}

% Télécharger un fichier 
\begin{figure}[H]
  \centering
  \includegraphics[width=0.7\textwidth]{out/diagrams/sprint4/download_file/download_file}
  \caption{Diagramme de séquence de cas d'utilisation « Télécharger un fichier   »}
  \label{fig:sequence_download_file}
\end{figure}
% Signer par glisser et déposer une image dans un fichier 
\begin{figure}[H]
  \centering
  \includegraphics[width=0.7\textwidth]{out/diagrams/sprint4/sign_by_image/sign_by_image}
  \caption{Diagramme de séquence de cas d'utilisation « Signer par glisser et déposer une image dans un fichier   »}
  \label{fig:sequence_sign_by_image}
\end{figure}

% Signer à main un fichier Modifier la position d’une signature d’un fichier 
\begin{figure}[H]
  \centering
  \includegraphics[width=0.7\textwidth]{out/diagrams/sprint4/sign_by_hand/sign_by_hand}
  \caption{Diagramme de séquence de cas d'utilisation « Signer à main un fichier   »}
  \label{fig:sequence_sign_by_hand}
\end{figure}
% Modifier la position d’une signature d’un fichier 
\begin{figure}[H]
  \centering
  \includegraphics[width=0.7\textwidth]{out/diagrams/sprint4/move_signature/move_signature}
  \caption{Diagramme de séquence de cas d'utilisation « Modifier la position d’une signature d’un fichier »}
  \label{fig:sequence_move_signature}
\end{figure}

% Supprimer une signature d’un fichier 
\begin{figure}[H]
  \centering
  \includegraphics[width=0.7\textwidth]{out/diagrams/sprint4/delete_signature/delete_signature}
  \caption{Diagramme de séquence de cas d'utilisation « Supprimer une signature d’un fichier »}
  \label{fig:sequence_move_signature}
\end{figure}
% Confirmer ou annuler les changements d’un fichier 
\begin{figure}[H]
  \centering
  \includegraphics[width=0.7\textwidth]{out/diagrams/sprint4/save_cancel_siganture/save_cancel_siganture}
  \caption{Diagramme de séquence de cas d'utilisation « Confirmer ou annuler les changements d’un fichier »}
  \label{fig:sequence_save_cancel_siganture}
\end{figure}



\subsubsection{Analyse détaillée}

\textbf{•	Diagramme de classe d'analyse de sprint "4" }

\begin{figure}[H]
  \centering
  \includegraphics[width=1\textwidth]{dca_sprint4}
  \caption{Diagramme de classe d'analyse de sprint 4}
  \label{fig:class_analyse_4}
\end{figure}



\subsubsection{Conception}

Après la présentation des diagrammes d'analyse, nous avons présenté dans cette partie les diagrammes de conception.
Nous allons présenter dans cette partie les diagrammes de conception de sprint 4.
\newpage 
\begin{landscape}

\textbf{•	Diagramme de classe de conception de sprint 4 : « Visualisation et signature d'un fichier »}

\begin{figure}[H]
  \centering
  \includegraphics[height=0.8\textheight]{dcc_spint4}
  \caption{Diagramme de classe de conception de sprint 4 : « Visualisation et signature d'un fichier »}
  \label{fig:class_diagram_4}
\end{figure}
\end{landscape}
\newpage


\begin{figure}[H]
  \centering
  \includegraphics[width=1\textwidth]{out/diagrams/sprint4/sequence_view_file/sequence_view_file}
  \caption{Diagramme de séquence de conception de cas d'utilisation « Visualiser un fichier »}
  \label{fig:sequence_conception_view_file}
\end{figure}

\begin{figure}[H]
  \centering
  \includegraphics[width=1\textwidth]{out/diagrams/sprint4/sequence_download_file/sequence_download_file}
  \caption{Diagramme de séquence de conception de cas d'utilisation « Télécharger un fichier »}
  \label{fig:sequence_conception_download_file}
\end{figure}

\begin{figure}[H]
  \centering
  \includegraphics[width=1\textwidth]{out/diagrams/sprint4/sequence_sign_by_image/sequence_sign_by_image}
  \caption{Diagramme de séquence de conception de cas d'utilisation « Signer par glisser et déposer une image dans un fichier »}
  \label{fig:sequence_conception_sign_by_image}
\end{figure}

\begin{figure}[H]
  \centering
  \includegraphics[width=1\textwidth]{out/diagrams/sprint4/sequence_sign_by_hand/sequence_sign_by_hand}
  \caption{Diagramme de séquence de conception de cas d'utilisation « Signer à main un fichier »}
  \label{fig:sequence_conception_sign_by_hand}
\end{figure}

\begin{figure}[H]
  \centering
  \includegraphics[width=1\textwidth]{out/diagrams/sprint4/sequence_move_signature/sequence_move_signature}
  \caption{Diagramme de séquence de conception de cas d'utilisation « Modifier la position d’une signature d’un fichier »}
  \label{fig:sequence_conception_move_signature}
\end{figure}

\begin{figure}[H]
  \centering
  \includegraphics[width=1\textwidth]{out/diagrams/sprint4/sequence_delete_signature/sequence_delete_signature}
  \caption{Diagramme de séquence de conception de cas d'utilisation « Supprimer une signature d’un fichier »}
  \label{fig:sequence_conception_delete_signature}
\end{figure}

\begin{figure}[H]
  \centering
  \includegraphics[width=1\textwidth]{out/diagrams/sprint4/sequence_save_cancel_siganture/sequence_save_cancel_siganture}
  \caption{Diagramme de séquence de conception de cas d'utilisation « Confirmer ou annuler les changements d’un fichier »}
  \label{fig:sequence_conception_save_cancel_siganture}
\end{figure}



\subsubsection{Réalisation}

\subsection{Sprint review:}


A la fin de ce sprint, nous avons planifié une autre réunion dans la société Neoledge  afin de vérifier notre démarche de travail par rapport au besoin de client tout en respectant le délai que nous avons prévu.

Nous avons fait une démonstration durant laquelle nous allons présenter notre incrément :
\begin{itemize}
  \item La visualisation d'un fichier.
  \item La signature à main d'un fichier.
  \item La signature par glisser et déposer une image dans un fichier.
  \item La modification de la position d'une signature d'un fichier.
  \item La suppression d'une signature d'un fichier.
  \item La confirmation ou l'annulation de la signature d'un fichier
\end{itemize}

\subsection{Sprint retrospective:}

Après la Sprint Review, nous avons réfléchi à des pistes pour améliorer la qualité et l'efficacité de notre application.


\noindent\textbf{•	Ce qui s'est bien passé :}
Nous avons terminé le sprint dans le délai.
\noindent\textbf{•	Ce qui s'est mal passé :}
\begin{itemize}
  \item Manque de documentation de l'ionic
  \item Difficulté tu développement du package neo-pdf-viewer et son intégration dans notre projet
  \item Difficultés de déposer la signature à la position exacte.
\end{itemize}

