\chapter*{Conclusion Générale et perspectives}
\addcontentsline{toc}{chapter}{Conclusion Générale et perspectives}
\markboth{Conclusion Générale et perspectives}{Conclusion Générale et perspectives}
\label{sec:conclusion}


Au cours de la réalisation de ce projet au sein de l'entreprise Neoledge pendant 4 mois, nous avons pu mettre en pratique nos connaissances acquises au cours de notre parcours universitaire. Notre mission consistait à concevoir et développer une application mobile responsive, utilisant le framework Ionic Vue, pour Elise, le produit phare de la société française NeoLedge. 

\medskip

D'un point de vue technique, nous avons eu l'opportunité d'utiliser divers langages de programmation tels que Vue JS, XML, JavaScript, Ionic et Capacitor. Cette expérience nous a permis de maîtriser le langage de modélisation UML et d'appliquer les concepts fondamentaux du framework de gestion de projet Scrum. Grâce à cette approche, nous avons cherché à adopter les meilleures solutions techniques et méthodes de développement. 

\medskip

Sur le plan personnel, cette expérience pratique et professionnelle nous a permis de découvrir le monde du travail avec ses exigences en matière de discipline et de responsabilité. Ces apprentissages seront précieux pour enrichir nos futurs parcours professionnels et nous aider à atteindre des niveaux d'excellence et de savoir plus élevés. 

\medskip

Parmi les défis rencontrés, nous avons notamment dû maîtriser la technologie Ionic et rechercher la meilleure solution pour développer un lecteur de fichiers dédié à Elise Mobile. 

\medskip

En ce qui concerne les perspectives d'amélioration, nous envisageons d'intégrer des fonctionnalités telles que la reconnaissance de signatures à partir d'une image scannée. De plus, nous pourrions envisager l'intégration de la fonctionnalité de signature numérique des documents à l'aide d'une carte RFID ou d'une clé USB. 

\medskip

Ce stage ne s'est pas limité à l'apprentissage de nouvelles technologies, il nous a également permis de développer des compétences essentielles telles que la ponctualité, le travail en équipe et l'adaptabilité face aux différents défis et contraintes du monde professionnel. 

\medskip

En conclusion, ce projet nous a offert une expérience enrichissante et nous a permis d'appliquer nos connaissances académiques dans un environnement professionnel concret. Nous remercions l'entreprise Neoledge pour cette opportunité et nous sommes convaincus que les compétences acquises lors de ce stage nous seront précieuses dans nos futures carrières. 